\newpage
\chapter{Введение}

\section{Криптосистемы с открытым ключом}
\paragraph{} \textit{Криптография} - область знаний, которая занимается разработкой методов преобразования информации с целью обеспечения ее 
конфиденциальности, целостности и аутентификации.

  Пусть \textit{A} и \textit{B}  - конечные множества, будем называть их алфавитами. Информацию, состоящую из конечного объединения элементов 
множества \textit{А}, которую будем защищать, будем называть \textit{открытым текстом}. Конечное объединение элементов множества \textit{B} будем 
называть \textit{шифротекстом}. Пусть \textit{X} и \textit{Y} - множества открытых текстов и шифрованых текстов соответственно.

  Функцию \textit{E\textsubscript{k} : \textit{X} {$\rightarrow$} \textit{Y}}, где \textit{k} - параметр функции,который будем называть ключом, 
принадледит множеству ключей \textit{K}, будем называть \textit{функцией шифрования}. 

  Функция \textit{D\textsubscript{k} : \textit{Y} {$\rightarrow$} \textit{X}}называется \textit{функцией дешифрования}.
  \textit{Шифром} или \textit{криптосистемой} называется набор (\textit{A, B, X, Y, K, E\textsubscript{k}, D\textsubscript{k}}), удовлетворяющий 
требованию \textit{D\textsubscript{k}(E\textsubscript{k}(x)) = x} для каждого \textit{x} {$\subseteq$} \textit{X} и \textit{k} {$\subseteq$} \textit{K}
  \textit{Шифрование} - процесс применения шифра к защищаемой информации, преобраование информации \textit{(открытого текста)} в шифрованное сообщение 
\textit{(шифротекст)} с помощью определенных правил, содержщихся в шифре.
  \textit{Дешифрование} - процесс, обратный \textit{шифрованию}, преоразрвание шифрованного сообщения в защищаемую информацию с помощью определенных 
правил, содержащихся в шифре.
  Криптосистемы (\textit{X, Y, K, E\textsubscript{k}, D\textsubscript{k}}), в которых в функции шифрования \textit{E\textsubscript{k}} и в функции 
дешифрования \textit{D\textsubscript{k}} используется один и тот же ключ \textit{k} {$\subseteq$} \textit{K}, называется симметричным. Шифры,в которых 
для штфрования используется один ключ, а для расщифрования - другой, называются ассиметричными или криптосистемами с открытым ключом. Таким образом, 
криптосистемой с открытым ключом называется система 
(\textit{X, Y, (k\textsubscript{e}, k\textsubscript{d}) {$\subseteq$} K, E\textsubscript{k\textsubscript{e}}, D\textsubscript{k\textsubscript{e},k\textsubscript{d}}}), 
где алгоритмы шифрования и дешифрования являются открытыми, шифрованный текст \textit{C} и открытый ключ \textit{k\textsubscript{e}} могут 
передаваться по незащищенному каналу, секретный ключ \textit{k\textsubscript{d}} является секретным.

Основные требования, которые предъялвяются к криптосистемам с открытым ключом:
\begin{enumerate}
	\item Вычисление пары (\textit{k\textsubscript{e}}, \textit{k\textsubscript{d}}) получателем должно быть простым (полиномиальный алгоритм).
	\item Отправитель, знаю открытый ключ \textit{k\textsubscript{e}} и сообщение \textit{m}, может легко вычислить криптограмму 
\textit{c = E\textsubscript{k\textsubscript{e}}(m)}.
	\item Получатель, используя секретный ключ \textit{k\textsubscript{d}} и криптограмму \textit{c}, может легко восстановить исходное сообщение 
\textit{m = D\textsubscript{k\textsubscript{d}}(c)}.
	\item Противник, зная открытый ключ \textit{k\textsubscript{e}}, при попытке вычислить секретный ключ \textit{k\textsubscript{d}} не может его 
вычилить
	\item Противник, зная пару (\textit{k\textsubscript{e}, c}), при попытке вычилить исходное сообщение \textit{m} не может его вычислить
\end{enumerate}

  В 1978 г. американцы Р. Ривест, А. Шамир и Л. Адлеман (R.L.Rivest. A.Shamir. L.Adleman) предложили пример функции \textit{f}, обладающей рядом 
замечательных достоинств. На её основе была построена реально используемая система шифрования, получившая название по первым буквам имен авторов - 
система RSA. Эта функция такова, что
\begin{enumerate}
    \item существует достаточно быстрый алгоритм вычисления значений \textit{f(x)};
    \item существует достаточно быстрый алгоритм вычисления значений обратной функции \textit{f\textsuperscript{ -1}(x)};
    \item функция \textit{f(x)} обладает некоторым «секретом», знание которого позволяет быстро вычислять значения \textit{f\textsuperscript{ -1}(x)};
в противном же случае вычисление \textit{f\textsuperscript{ -1}(x)} становится трудно разрешимой в вычислительном отношении задачей, требующей для 
своего решения столь много времени, что по его прошествии зашифрованная информация перестает представлять интерес для лиц, 
использующих отображение \textit{f} в качестве шифра.
\end{enumerate}	

\section{О криптосистеме RSA}

\paragraph{} Пусть \textit{n} и \textit{e} натуральные числа. Функция \textit{f} реализующая схему RSA, устроена следующим образом
\begin{equation}
  \textit{f : x {$\rightarrow$} x\textsuperscript{e} (mod n)},
\end{equation}
Для расшифровки сообщения \textit{a = f(x)} достаточно решить сравнение 

\begin{equation}
  \textit{x\textsuperscript{e} = a (mod n)} 
\end{equation}
При некоторых условиях на \textit{n} и \textit{e} это сравнение имеет единственное решение \textit{x}.

  Для того, чтобы описать эти условия и объяснить, как можно найти решение, нам потребуется одна теоретико-числовая функция - функция Эйлера. 
Эта функция натурального аргумента \textit{n} обозначается \textit{{$\varphi$}(n)} и равняется количеству целых чисел на отрезке от 1 до \textit{n}, 
взаимно простых с \textit{n}. Так \textit{{$\varphi$}(1) = 1} и \textit{{$\varphi$}(p\textsuperscript{ r}) = p\textsuperscript{ r - 1}(p - 1)} 
для любого простого числа \textit{p} и натурального \textit{r}. Кроме того, \textit{{$\varphi$}(a b) = {$\varphi$}(b) {$\varphi$}(a)} 
для любых натуральных взаимно простых \textit{a} и \textit{b}. Эти свойства позволяют легко вычислить значение \textit{{$\varphi$}(n)}, если известно 
разложение числа \textit{n} на простые сомножители. 

  Если показатель степени \textit{e} в сравнении (3.1.2) взаимно прост с \textit{{$\varphi$}(n)}, то сравнение (3.1.2) имеет единственное решение. 
Для того, чтобы найти его, определим целое число \textit{d}, удовлетворяющее условиям. 
\begin{equation}
 \textit{d e {$\equiv$} (mod {$\varphi$}(n)), 1 {$\leq$} d {$<$} {$\varphi$}}.
\end{equation}
Такое число существует, поскольку \textit{(e, {$\varphi$}(n)) = 1}, и притом единственно. Здесь и далее символом \textit{(a, b)} будет обозначаться 
наибольший общий делитель чисел \textit{a} и \textit{b}. Классическая теорема Эйлера, утверждает, что для каждого числа \textit{x}, взаимно простого 
с \textit{n}, выполняется сравнение \textit{x\textsuperscript{ {$\varphi$}(n)} {$\equiv$} 1 (mod n) } и, следовательно
\begin{equation}
 \textit{a\textsuperscript{ d} {$\equiv$} x\textsuperscript{ d e} {$\equiv$} x (mod n)}.
\end{equation}
Таким образом, в предположении \textit{(a, m) = 1}, единственное решение сравнения (3.1.2) может быть найдено в виде
\begin{equation}
 \textit{x {$\equiv$} a\textsuperscript{ d} (mod n)}.
\end{equation}
Если дополнительно предположить, что число \textit{n} состоит из различных простых сомножителей, то сравнение (3.5) будет выполняться и без 
предположения \textit{(a, m) = 1}. Действительно, обозначим \textit{r = (a, n)} и \textit{s = {$\frac{n}{r}$}}. Тогда {$\varphi$}(n) делится на {$\varphi$}(r), 
а из (3.2) следует, что \textit{(x, s) = 1}. Подобно (3.4), теперь легко находим (3.5). А кроме того, имеем \textit{x {$\equiv$} 0 {$\equiv$} a\textsuperscript{ r} (mod r)}. 
Получившиеся сравнения в силу \textit{(r, s) = 1} дают нам (3.5).

  Функция (3.1), принятая в системе RSA, может быть вычислена достаточно быстро. Обратная к \textit{f(x)} функция 
\textit{f\textsuperscript{ -1} : x {$\rightarrow$} x\textsuperscript{ d} (mod n) } вычисляется по тем же правилам, что и \textit{f(x)}, 
лишь с заменой показателя степени \textit{e} на \textit{d}.
\begin{comment}
  Для вычисления функции (3.1) достаточно знать лишь числа \textit{e} и \textit{n}. Именно они составляют открытый ключ для шифрования. 
А вот для вычисления обратной функции требуется знать число \textit{d}. Казалось бы, ничего не стоит, зная число \textit{n}, разложить 
его на простые сомножители, вычислить затем с помощью известных правил значение \textit{{$\varphi$}(n)} и, наконец, с помощью (3.3) определить 
нужное число \textit{d}. Все шаги этого вычисления могут быть реализованы достаточно быстро, за исключением первого. Именно разложение числа \textit{n} на 
простые множители и составляет наиболее трудоемкую часть вычислений. В теории чисел несмотря на многолетнюю её историю и на очень интенсивные поиски в течение последних 20 лет, 
эффективный алгоритм разложения натуральных чисел на множители так и не найден.
\end{comment}
  Авторы схемы RSA предложили выбирать число \textit{n} в виде произведения двух простых множителей \textit{p} и \textit{q}, примерно одинаковых по 
величине. Так как 
\begin{equation}
 \textit{{$\varphi$}(n) = {$\varphi$}(p q) = (p-1)(q-1)},
\end{equation}
то единственное условие на выбор показателя степени \textit{e} в отображении (1) есть
\begin{equation}
 \textit{(e, p - 1) = (e, q - 1) = 1}
\end{equation}

  Итак, лицо, заинтересованное в организации шифрованной переписки с помощью схемы RSA, выбирает два достаточно больших простых числа \textit{p} и \textit{q}. 
Перемножая их, оно находит число \textit{n = p q}. Затем выбирается число \textit{e}, удовлетворяющее условиям (3.7), вычисляется с помощью (3.6) 
число \textit{{$\varphi$}(n)} и с помощью (3.3) - число \textit{d}. Числа \textit{n} и \textit{e} публикуются, число \textit{d} остается секретным.

\begin{comment}
\paragraph{} Труды Евклида и Диофанта, Ферма и Эйлера, Гаусса, Чебышева и Эрмита содержат остроумные и весьма эффективные алгоритмы решения диофантовых 
уравнений, выяснения разрешимости сравнений, построения больших по тем временам простых чисел, нахождения наилучших приближений и т.д. В последние два 
десятилетия, благодаря в первую очередь запросам криптографии и широкому распространению ЭВМ, исследования по алгоритмическим вопросам теории чисел 
переживают период бурного и весьма плодотворного развития. Вычислительные машины и электронные средства связи проникли практически во все сферы 
человеческой деятельности. Немыслима без них и современная криптография. Шифрование и дешифрование текстов можно представлять себе как процессы 
переработки целых чисел при помощи ЭВМ, а способы, которыми выполняются эти операции, как некоторые функции, определённые на множестве целых чисел. 
Всё это делает естественным появление в криптографии методов теории чисел. Кроме того, стойкость ряда современных криптосистем обосновывается только 
сложностью некоторых теоретико-числовых задач. Но возможности ЭВМ имеют определённые границы. Приходится разбивать длинную цифровую последовательность 
на блоки ограниченной длины и шифровать каждый такой блок отдельно. Мы будем считать в дальнейшем, что все шифруемые целые числа неотрицательны и по 
величине меньше некоторого заданного (скажем, техническими ограничениями) числа \textit{m}. Таким же условиям будут удовлетворять и числа, получаемые 
в процессе шифрования. Это позволяет считать и те, и другие числа элементами кольца вычетов. Шифрующая функция при этом может рассматриваться как 
взаимнооднозначное отображение колец вычетов а число  представляет собой сообщение  в зашифрованном виде.

  Простейший шифр такого рода - шифр замены, соответствует отображению
\begin{equation}
    \textit{f : x {$\rightarrow$} x + k (mod m)}
\end{equation}
при некотором фиксированном целом \textit{k}. Подобный шифр использовал еще Юлий Цезарь. Конечно, не каждое отображение  подходит для целей надежного 
сокрытия информации.
\end{comment}

\begin{comment}
  Еще до выхода из печати статьи копия доклада в Массачусетском Технологическом институте, посвящённого системе RSA. была послана известному 
популяризатору математики М. Гарднеру, который в 1977 г. в журнале Scientific American опубликовал статью посвящённую этой системе шифрования. 
В русском переводе заглавие статьи Гарднера звучит так: Новый вид шифра, на расшифровку которого потребуются миллионы лет. Именно эта статья сыграла 
важнейшую роль в распространении информации об RSA, привлекла к криптографии внимание широких кругов неспециалистов и фактически способствовала 
бурному прогрессу этой области, произошедшему в последовавшие 20 лет.
\end{comment}
\section{Постановка задачи}