\newpage
\chapter{Введение}

\section{Основные определения}

\paragraph{} Криптография — наука о методах обеспечения конфиденциальности (невозможности прочтения информации посторонним) и аутентичности (целостности и подлинности авторства, а также невозможности отказа от авторства) информации.

Изначально криптография изучала методы шифрования информации — обратимого преобразования открытого (исходного) текста на основе секретного алгоритма и/или ключа в шифрованный текст (\textit{шифротекст}). Традиционная криптография образует раздел симметричных криптосистем, в которых зашифрование и расшифрование проводится с использованием одного и того же \textit{секретного ключа}. Помимо этого раздела современная криптография включает в себя асимметричные криптосистемы, системы электронной цифровой подписи (ЭЦП), хеш-функции, управление ключами, получение скрытой информации, квантовую криптографию.

В традиционном шифровании с \textit{секретным ключом} (secret key) (симметричное шифрование) зашифровывающий и расшифровывающий ключи, совпадают. Стороны, обменивающиеся зашифрованными данными, должны знать общий секретный ключ. Процесс обмена информацией о секретном ключе представляет собой брешь в безопасности вычислительной системы.

Фундаментальное отличие шифрования с \textit{открытым ключом} (асимметричное шифрование) заключается в том, что зашифровывающий и расшифровывающий ключи не совпадают. Шифрование информации является односторонним процессом: открытые данные шифруются с помощью зашифровывающего ключа, однако с помощью того же ключа нельзя осуществить обратное преобразование и получить открытые данные. Для этого необходим расшифровывающий ключ, который связан с зашифровывающим ключом, но не совпадает с ним. Подобная технология шифрования предполагает, что каждый пользователь имеет в своем распоряжении пару ключей — \textit{открытый ключ} (public key) и личный или \textit{закрытый ключ} (private key). Свободно распространяя \textit{открытый ключ}, вы даете возможность другим пользователям посылать вам зашифрованные данные, которые могут быть расшифрованы с помощью известного только вам личного ключа. Аналогично, с помощью личного ключа вы можете преобразовать данные так, чтобы другая сторона убедилась в том, что информация пришла именно от вас. Эта возможность применяется при работе с цифровыми или электронными подписями. Шифрование с \textit{открытым ключом} имеет все возможности шифрования с \textit{закрытым ключом}, но может проходить медленнее из-за необходимости генерировать два ключа. Однако этот метод безопаснее.

\section{Строгие математические определения}

\paragraph{} \textit{Криптография} - область знаний, которая занимается разработкой методов преобразования информации с целью обеспечения ее конфиденциальности, целостности и аутентификации.

Пусть \textit{A} и \textit{B}  - конечные множества, будем называть их алфавитами. Информацию, состоящую из конечного объединения элементов множества \textit{А}, которую будем защищать, будем называть \textit{открытым текстом}. Конечное объединение элементов множества \textit{B} будем называть \textit{шифротекстом}. Пусть \textit{X} и \textit{Y} - множества открытых текстов и шифрованых текстов соответственно.

Функцию \textit{E\textsubscript{k} : \textit{X} {$\rightarrow$} \textit{Y}}, где \textit{k} - параметр функции,который будем называть ключом, принадледит множеству ключей \textit{K}, будем называть \textit{функцией шифрования}. Функция \textit{D\textsubscript{k} : \textit{Y} {$\rightarrow$} \textit{X}}называется \textit{функцией дешифрования}.

\textit{Шифром} или \textit{криптосистемой} называется набор (\textit{A, B, X, Y, K, E\textsubscript{k}, D\textsubscript{k}}), удовлетворяющий требованию \textit{D\textsubscript{k}(E\textsubscript{k}(x)) = x} для каждого \textit{x} {$\subseteq$} \textit{X} и \textit{k} {$\subseteq$} \textit{K}

\textit{Шифрование} - процесс применения шифра к защищаемой информации, преобраование информации \textit{(открытого текста)} в шифрованное сообщение \textit{(шифротекст)} с помощью определенных правил, содержщихся в шифре.

\textit{Дешифрование} - процесс, обратный \textit{шифрованию}, преоразрвание шифрованного сообщения в защищаемую информацию с помощью определенных правил, содержащихся в шифре.

Криптосистемы (\textit{X, Y, K, E\textsubscript{k}, D\textsubscript{k}}), в которых в функции шифрования \textit{E\textsubscript{k}} и в функции дешифрования \textit{D\textsubscript{k}} используется один и тот же ключ \textit{k} {$\subseteq$} \textit{K}, называется симметричным. Шифры,в которых для штфрования используется один ключ, а для расщифрования - другой, называются ассиметричными или криптосистемами с открытым ключом. Таким образом, криптосистемой с открытым ключом называется система (\textit{X, Y, (k\textsubscript{e}, k\textsubscript{d}) {$\subseteq$} K, E\textsubscript{k\textsubscript{e}}, D\textsubscript{k\textsubscript{e},k\textsubscript{d}}}), где алгоритмы шифрования и дешифрования являются открытыми, шифрованный текст \textit{C} и открытый ключ \textit{k\textsubscript{e}} могут передаваться по незащищенному каналу, секретный ключ \textit{k\textsubscript{d}} является секретным.

Основные требования, которые предъялвяются к криптосистемам с открытым ключом:
\begin{enumerate}
	\item Вычисление пары (\textit{k\textsubscript{e}}, \textit{k\textsubscript{d}}) получателем должно быть простым (полиномиальный алгоритм).
	\item Отправитель, знаю открытый ключ \textit{k\textsubscript{e}} и сообщение \textit{m}, может легко вычислить криптограмму \textit{c = E\textsubscript{k\textsubscript{e}}(m)}.
	\item Получатель, используя секретный ключ \textit{k\textsubscript{d}} и криптограмму \textit{c}, может легко восстановить исходное сообщение \textit{m = D\textsubscript{k\textsubscript{d}}(c)}.
	\item Противник, зная открытый ключ \textit{k\textsubscript{e}}, при попытке вычислить секретный ключ \textit{k\textsubscript{d}} не может его высилить
	\item Противник, зная пару (\textit{k\textsubscript{e}, c}), при попытке вычилить исходное сообщение \textit{m} не может его вычислить
\end{enumerate}

\section{Постановка задачи}