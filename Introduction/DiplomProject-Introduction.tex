\newpage
\chapter{Введение}

\section{Основные определения}
\paragraph{} \textit{Криптография} - область знаний, которая занимается разработкой методов преобразования информации с целью обеспечения ее 
конфиденциальности, целостности и аутентификации.

  Пусть \textit{A} и \textit{B}  - конечные множества, будем называть их алфавитами. Информацию, состоящую из конечного объединения элементов 
множества \textit{А}, которую будем защищать, будем называть \textit{открытым текстом}. Конечное объединение элементов множества \textit{B} будем 
называть \textit{шифротекстом}. Пусть \textit{X} и \textit{Y} - множества открытых текстов и шифрованых текстов соответственно.

  Функцию \textit{E\textsubscript{k} : \textit{X} {$\rightarrow$} \textit{Y}}, где \textit{k} - параметр функции,который будем называть ключом, 
принадледит множеству ключей \textit{K}, будем называть \textit{функцией шифрования}. 

  Функция \textit{D\textsubscript{k} : \textit{Y} {$\rightarrow$} \textit{X}}называется \textit{функцией дешифрования}.
  \textit{Шифром} или \textit{криптосистемой} называется набор (\textit{A, B, X, Y, K, E\textsubscript{k}, D\textsubscript{k}}), удовлетворяющий 
требованию \textit{D\textsubscript{k}(E\textsubscript{k}(x)) = x} для каждого \textit{x} {$\subseteq$} \textit{X} и \textit{k} {$\subseteq$} \textit{K}
  \textit{Шифрование} - процесс применения шифра к защищаемой информации, преобраование информации \textit{(открытого текста)} в шифрованное сообщение 
\textit{(шифротекст)} с помощью определенных правил, содержщихся в шифре.
  \textit{Дешифрование} - процесс, обратный \textit{шифрованию}, преоразрвание шифрованного сообщения в защищаемую информацию с помощью определенных 
правил, содержащихся в шифре.
  Криптосистемы (\textit{X, Y, K, E\textsubscript{k}, D\textsubscript{k}}), в которых в функции шифрования \textit{E\textsubscript{k}} и в функции 
дешифрования \textit{D\textsubscript{k}} используется один и тот же ключ \textit{k} {$\subseteq$} \textit{K}, называется симметричным. Шифры,в которых 
для штфрования используется один ключ, а для расщифрования - другой, называются ассиметричными или криптосистемами с открытым ключом. Таким образом, 
криптосистемой с открытым ключом называется система 
(\textit{X, Y, (k\textsubscript{e}, k\textsubscript{d}) {$\subseteq$} K, E\textsubscript{k\textsubscript{e}}, D\textsubscript{k\textsubscript{e},k\textsubscript{d}}}), 
где алгоритмы шифрования и дешифрования являются открытыми, шифрованный текст \textit{C} и открытый ключ \textit{k\textsubscript{e}} могут 
передаваться по незащищенному каналу, секретный ключ \textit{k\textsubscript{d}} является секретным.

Основные требования, которые предъялвяются к криптосистемам с открытым ключом:
\begin{enumerate}
	\item Вычисление пары (\textit{k\textsubscript{e}}, \textit{k\textsubscript{d}}) получателем должно быть простым (полиномиальный алгоритм).
	\item Отправитель, знаю открытый ключ \textit{k\textsubscript{e}} и сообщение \textit{m}, может легко вычислить криптограмму 
\textit{c = E\textsubscript{k\textsubscript{e}}(m)}.
	\item Получатель, используя секретный ключ \textit{k\textsubscript{d}} и криптограмму \textit{c}, может легко восстановить исходное сообщение 
\textit{m = D\textsubscript{k\textsubscript{d}}(c)}.
	\item Противник, зная открытый ключ \textit{k\textsubscript{e}}, при попытке вычислить секретный ключ \textit{k\textsubscript{d}} не может его 
вычилить
	\item Противник, зная пару (\textit{k\textsubscript{e}, c}), при попытке вычилить исходное сообщение \textit{m} не может его вычислить
\end{enumerate}

\section{Введение в теорию шифрования}

\paragraph{} Труды Евклида и Диофанта, Ферма и Эйлера, Гаусса, Чебышева и Эрмита содержат остроумные и весьма эффективные алгоритмы решения диофантовых 
уравнений, выяснения разрешимости сравнений, построения больших по тем временам простых чисел, нахождения наилучших приближений и т.д. В последние два 
десятилетия, благодаря в первую очередь запросам криптографии и широкому распространению ЭВМ, исследования по алгоритмическим вопросам теории чисел 
переживают период бурного и весьма плодотворного развития. Вычислительные машины и электронные средства связи проникли практически во все сферы 
человеческой деятельности. Немыслима без них и современная криптография. Шифрование и дешифрование текстов можно представлять себе как процессы 
переработки целых чисел при помощи ЭВМ, а способы, которыми выполняются эти операции, как некоторые функции, определённые на множестве целых чисел. 
Всё это делает естественным появление в криптографии методов теории чисел. Кроме того, стойкость ряда современных криптосистем обосновывается только 
сложностью некоторых теоретико-числовых задач. Но возможности ЭВМ имеют определённые границы. Приходится разбивать длинную цифровую последовательность 
на блоки ограниченной длины и шифровать каждый такой блок отдельно. Мы будем считать в дальнейшем, что все шифруемые целые числа неотрицательны и по 
величине меньше некоторого заданного (скажем, техническими ограничениями) числа \textit{m}. Таким же условиям будут удовлетворять и числа, получаемые 
в процессе шифрования. Это позволяет считать и те, и другие числа элементами кольца вычетов. Шифрующая функция при этом может рассматриваться как 
взаимнооднозначное отображение колец вычетов а число  представляет собой сообщение  в зашифрованном виде.

  Простейший шифр такого рода - шифр замены, соответствует отображению
\begin{equation}
    \textit{f : x {$\rightarrow$} x + k (mod m)}
\end{equation}
при некотором фиксированном целом \textit{k}. Подобный шифр использовал еще Юлий Цезарь. Конечно, не каждое отображение  подходит для целей надежного 
сокрытия информации.

  В 1978 г. американцы Р. Ривест, А. Шамир и Л. Адлеман (R.L.Rivest. A.Shamir. L.Adleman) предложили пример функции \textit{f}, обладающей рядом 
замечательных достоинств. На её основе была построена реально используемая система шифрования, получившая название по первым буквам имен авторов - 
система RSA. Эта функция такова, что
\begin{enumerate}
    \item существует достаточно быстрый алгоритм вычисления значений \textit{f(x)};
    \item существует достаточно быстрый алгоритм вычисления значений обратной функции \textit{f\textsuperscript{ -1}(x)};
    \item функция \textit{f(x)} обладает некоторым «секретом», знание которого позволяет быстро вычислять значения \textit{f\textsuperscript{ -1}(x)};
в противном же случае вычисление \textit{f\textsuperscript{ -1}(x)} становится трудно разрешимой в вычислительном отношении задачей, требующей для 
своего решения столь много времени, что по его прошествии зашифрованная информация перестает представлять интерес для лиц, 
использующих отображение \textit{f} в качестве шифра.
\end{enumerate}	

  Еще до выхода из печати статьи копия доклада в Массачусетском Технологическом институте, посвящённого системе RSA. была послана известному 
популяризатору математики М. Гарднеру, который в 1977 г. в журнале Scientific American опубликовал статью посвящённую этой системе шифрования. 
В русском переводе заглавие статьи Гарднера звучит так: Новый вид шифра, на расшифровку которого потребуются миллионы лет. Именно эта статья сыграла 
важнейшую роль в распространении информации об RSA, привлекла к криптографии внимание широких кругов неспециалистов и фактически способствовала 
бурному прогрессу этой области, произошедшему в последовавшие 20 лет.

\section{Постановка задачи}