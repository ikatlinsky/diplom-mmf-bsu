% !TEX root = ../DiplomProject.tex

\newpage
\chapter*{Введение}
\addcontentsline{toc}{chapter}{Введение}

\section*{Криптосистемы с открытым ключом}
\addcontentsline{toc}{section}{Криптосистемы с открытым ключом}
\paragraph{} \textit{Криптография} - область знаний, которая занимается разработкой методов преобразования информации с целью обеспечения ее 
конфиденциальности, целостности и аутентификации.

  Пусть $A$ и $B$  - конечные множества, будем называть их алфавитами. Информацию, состоящую из конечного объединения элементов 
множества $A$, которую будем защищать, будем называть \textit{открытым текстом}. Конечное объединение элементов множества $B$ будем 
называть \textit{шифротекстом}. Пусть $X$ и $Y$ - множества открытых текстов и шифрованых текстов соответственно.

  Функцию $E_{k} : X \rightarrow Y$, где $k$ - параметр функции,который будем называть ключом, 
принадледит множеству ключей $K$, будем называть \textit{функцией шифрования}. 

  Функция $D_{k} : Y \rightarrow X$ называется \textit{функцией дешифрования}.

  \textit{Шифром} или \textit{криптосистемой} называется набор 
  \begin{equation}
    (A, B, X, Y, K, E_{k}, D_{k}),
  \end{equation}
  удовлетворяющий требованию $D_{k}(E_{k}(x)) = x$ для каждого $x \subseteq X$ и $k \subseteq K$.

  \textit{Шифрование} - процесс применения шифра к защищаемой информации, преобраование информации \textit{(открытого текста)} в шифрованное сообщение \textit{(шифротекст)} с помощью определенных правил, содержщихся в шифре.

  \textit{Дешифрование} - процесс, обратный \textit{шифрованию}, преоразрвание шифрованного сообщения в защищаемую информацию с помощью определенных 
правил, содержащихся в шифре.

  Криптосистемы ($X, Y, K, E_{k}, D_{k}$), в которых в функции шифрования $E_{k}$ и в функции 
дешифрования $D_{k}$ используется один и тот же ключ $k \subseteq K$, называется симметричным. 

  Шифры,в которых для шифрования используется один ключ, а для расшифрования - другой, называются ассиметричными или криптосистемами с открытым ключом. Таким образом, криптосистемой с открытым ключом называется система 
  \begin{equation}
    (X, Y, (k_{e}, k_{d}) \subseteq K, E_{k_{e}}, D_{k_{e},k_{d}}), 
  \end{equation}
  где алгоритмы шифрования и дешифрования являются открытыми, шифрованный текст $C$ и открытый ключ $k_{e}$ могут 
передаваться по незащищенному каналу, секретный ключ \textit{k\textsubscript{d}} является секретным.

Основные требования, которые предъялвяются к криптосистемам с открытым ключом:
\begin{enumerate}
	\item Вычисление пары ($k_{e}, k_{d}$) получателем должно быть простым (полиномиальный алгоритм).
	\item Отправитель, знаю открытый ключ $k_{e}$ и сообщение $m$, может легко вычислить криптограмму 
$c = E_{k_{e}}(m)$.
	\item Получатель, используя секретный ключ $k_{d}$ и криптограмму $c$, может легко восстановить исходное сообщение 
$m = D_{k_{d}}(c)$.
	\item Противник, зная открытый ключ $k_{e}$, при попытке вычислить секретный ключ $k_{d}$ не может его 
вычилить
	\item Противник, зная пару ($k_{e}, c$), при попытке вычилить исходное сообщение $m$ не может его вычислить
\end{enumerate}

  В 1978 г. американцы Р. Ривест, А. Шамир и Л. Адлеман (R.L.Rivest. A.Shamir. L.Adleman) предложили пример функции $f$, обладающей рядом 
замечательных достоинств. На её основе была построена реально используемая система шифрования, получившая название по первым буквам имен авторов - 
система RSA. Эта функция такова, что
\begin{enumerate}
    \item существует достаточно быстрый алгоритм вычисления значений $f(x)$;
    \item существует достаточно быстрый алгоритм вычисления значений обратной функции $f^{ -1}(x)$;
    \item функция $f(x)$ обладает некоторым «секретом», знание которого позволяет быстро вычислять значения $f^{ -1}(x)$;
в противном же случае вычисление $f^{ -1}(x)$ становится трудно разрешимой в вычислительном отношении задачей, требующей для 
своего решения столь много времени, что по его прошествии зашифрованная информация перестает представлять интерес для лиц, 
использующих отображение \textit{f} в качестве шифра.
\end{enumerate}	

\section*{О криптосистеме RSA}
\addcontentsline{toc}{section}{О криптосистеме RSA}

\paragraph{} Пусть $n$ и $e$ натуральные числа. Функция $f$ реализующая схему RSA, устроена следующим образом
\begin{equation} \label{eq:intro-rsa-f}
  f : x \rightarrow x^{e} (mod \: n),
\end{equation}
Для расшифровки сообщения $a = f(x)$ достаточно решить сравнение 

\begin{equation} \label{eq:intro-rsa-xe}
  x^{e} = a (mod \: n). 
\end{equation}

При некоторых условиях на $n$ и $e$ это сравнение имеет единственное решение $x$.

  Для того, чтобы описать эти условия и объяснить, как можно найти решение, нам потребуется одна теоретико-числовая функция - функция Эйлера. 
Эта функция натурального аргумента $n$ обозначается $\varphi(n)$ и равняется количеству целых чисел на отрезке от 1 до $n$, 
взаимно простых с $n$. Так $\varphi(1) = 1$ и $\varphi(p^{ r}) = p^{ r - 1}(p - 1)$ 
для любого простого числа $p$ и натурального $r$. Кроме того, $\varphi(a b) = \varphi(b) \varphi(a)$ 
для любых натуральных взаимно простых $a$ и $b$. Эти свойства позволяют легко вычислить значение $\varphi(n)$, если известно 
разложение числа $n$ на простые сомножители. 

  Если показатель степени $e$ в сравнении (\ref{eq:intro-rsa-xe}) взаимно прост с $\varphi(n)$, то сравнение (\ref{eq:intro-rsa-xe}) имеет единственное решение. 
Для того, чтобы найти его, определим целое число $d$, удовлетворяющее условиям. 
\begin{equation} \label{eq:intro-rsa-de}
 d e \equiv (mod \: \varphi(n)), 1 \leq d < \varphi.
\end{equation}
Такое число существует, поскольку $(e, \varphi(n)) = 1$, и притом единственно. Здесь и далее символом $(a, b)$ будет обозначаться 
наибольший общий делитель чисел $a$ и $b$. Классическая теорема Эйлера, утверждает, что для каждого числа $x$, взаимно простого 
с $n$, выполняется сравнение $x^{\varphi(n)} \equiv 1 (mod \: n)$ и, следовательно
\begin{equation} \label{eq:intro-rsa-ad}
 a^{ d} \equiv x^{ d e} \equiv x (mod \: n).
\end{equation}
Таким образом, в предположении $(a, m) = 1$, единственное решение сравнения (\ref{eq:intro-rsa-xe}) может быть найдено в виде
\begin{equation} \label{eq:intro-rsa-xa}
 x \equiv a^{ d} (mod \: n).
\end{equation}
Если дополнительно предположить, что число $n$ состоит из различных простых сомножителей, то сравнение (\ref{eq:intro-rsa-xa}) будет выполняться и без 
предположения $(a, m) = 1$. Действительно, обозначим $r = (a, n)$ и $s = \frac{n}{r}$. Тогда $\varphi(n)$ делится на $\varphi(r)$, 
а из (\ref{eq:intro-rsa-xe}) следует, что $(x, s) = 1$. Подобно (\ref{eq:intro-rsa-ad}), теперь легко находим (\ref{eq:intro-rsa-xa}). А кроме того, имеем $x \equiv 0 \equiv a^{ r} (mod \: r)$. 
Получившиеся сравнения в силу $(r, s) = 1$ дают нам (\ref{eq:intro-rsa-xa}).

  Функция (\ref{eq:intro-rsa-f}), принятая в системе RSA, может быть вычислена достаточно быстро. Обратная к $f(x)$ функция 
$f^{ -1} : x \rightarrow x^{ d} (mod \: n)$ вычисляется по тем же правилам, что и $f(x)$, 
лишь с заменой показателя степени $e$ на $d$.
\begin{comment}
  Для вычисления функции (\ref{eq:intro-rsa-f}) достаточно знать лишь числа \textit{e} и \textit{n}. Именно они составляют открытый ключ для шифрования. 
А вот для вычисления обратной функции требуется знать число \textit{d}. Казалось бы, ничего не стоит, зная число \textit{n}, разложить 
его на простые сомножители, вычислить затем с помощью известных правил значение \textit{{$\varphi$}(n)} и, наконец, с помощью (3.3) определить 
нужное число \textit{d}. Все шаги этого вычисления могут быть реализованы достаточно быстро, за исключением первого. Именно разложение числа \textit{n} на 
простые множители и составляет наиболее трудоемкую часть вычислений. В теории чисел несмотря на многолетнюю её историю и на очень интенсивные поиски в течение последних 20 лет, 
эффективный алгоритм разложения натуральных чисел на множители так и не найден.
\end{comment}
  Авторы схемы RSA предложили выбирать число $n$ в виде произведения двух простых множителей $p$ и $q$, примерно одинаковых по 
величине. Так как 

\begin{equation} \label{eq:intro-rsa-fin}
 \varphi(n) = \varphi(p q) = (p-1)(q-1),
\end{equation}

то единственное условие на выбор показателя степени $e$ в отображении (1) есть

\begin{equation} \label{eq:intro-rsa-ep}
 (e, p - 1) = (e, q - 1) = 1
\end{equation}

  Итак, лицо, заинтересованное в организации шифрованной переписки с помощью схемы RSA, выбирает два достаточно больших простых числа $p$ и $q$. 
Перемножая их, оно находит число $n = p q$. Затем выбирается число $e$, удовлетворяющее условиям (\ref{eq:intro-rsa-ep}), вычисляется с помощью (\ref{eq:intro-rsa-fin}) 
число $\varphi(n)$ и с помощью (\ref{eq:intro-rsa-de}) - число $d$. Числа $n$ и $e$ публикуются, число $d$ остается секретным.

\begin{comment}
\paragraph{} Труды Евклида и Диофанта, Ферма и Эйлера, Гаусса, Чебышева и Эрмита содержат остроумные и весьма эффективные алгоритмы решения диофантовых 
уравнений, выяснения разрешимости сравнений, построения больших по тем временам простых чисел, нахождения наилучших приближений и т.д. В последние два 
десятилетия, благодаря в первую очередь запросам криптографии и широкому распространению ЭВМ, исследования по алгоритмическим вопросам теории чисел 
переживают период бурного и весьма плодотворного развития. Вычислительные машины и электронные средства связи проникли практически во все сферы 
человеческой деятельности. Немыслима без них и современная криптография. Шифрование и дешифрование текстов можно представлять себе как процессы 
переработки целых чисел при помощи ЭВМ, а способы, которыми выполняются эти операции, как некоторые функции, определённые на множестве целых чисел. 
Всё это делает естественным появление в криптографии методов теории чисел. Кроме того, стойкость ряда современных криптосистем обосновывается только 
сложностью некоторых теоретико-числовых задач. Но возможности ЭВМ имеют определённые границы. Приходится разбивать длинную цифровую последовательность 
на блоки ограниченной длины и шифровать каждый такой блок отдельно. Мы будем считать в дальнейшем, что все шифруемые целые числа неотрицательны и по 
величине меньше некоторого заданного (скажем, техническими ограничениями) числа \textit{m}. Таким же условиям будут удовлетворять и числа, получаемые 
в процессе шифрования. Это позволяет считать и те, и другие числа элементами кольца вычетов. Шифрующая функция при этом может рассматриваться как 
взаимнооднозначное отображение колец вычетов а число  представляет собой сообщение  в зашифрованном виде.

  Простейший шифр такого рода - шифр замены, соответствует отображению
\begin{equation}
    \textit{f : x {$\rightarrow$} x + k (mod m)}
\end{equation}
при некотором фиксированном целом \textit{k}. Подобный шифр использовал еще Юлий Цезарь. Конечно, не каждое отображение  подходит для целей надежного 
сокрытия информации.
\end{comment}

\begin{comment}
  Еще до выхода из печати статьи копия доклада в Массачусетском Технологическом институте, посвящённого системе RSA. была послана известному 
популяризатору математики М. Гарднеру, который в 1977 г. в журнале Scientific American опубликовал статью посвящённую этой системе шифрования. 
В русском переводе заглавие статьи Гарднера звучит так: Новый вид шифра, на расшифровку которого потребуются миллионы лет. Именно эта статья сыграла 
важнейшую роль в распространении информации об RSA, привлекла к криптографии внимание широких кругов неспециалистов и фактически способствовала 
бурному прогрессу этой области, произошедшему в последовавшие 20 лет.
\end{comment}

\section*{Постановка задачи}
\addcontentsline{toc}{section}{Постановка задачи}

  \paragraph{} Целью выполнения данной дипломной работы является изучение алгоритмов построения и поиска больших простых чисел, алгоритмов факторизации и анализ 
  шифра RSA.

    Для выполнения поставленных целей необходимо успешно выполнить следующие задачи:

    \begin{enumerate}
      \item Изучить способы построения больших простых чисел - рассмотреть несколько различных способов получения простых чисел.
      \item Изучить и провести сравнение некоторых алгоритмов факторизации - рассмотреть примеры экспоненциальных и субэкспоненциальных алгоритмов.
      \item Провести криптоанализ шифра RSA - привести примеры атак на шифр, а также указать уязвимые места при использовании шифра RSA.
      \item Привести пример практического использования и практической реализации шифра RSA - рассмотреть применение RSA на примере RSA-OAEP.
      \item Систематизировать изученный материал и предложить рекомендации по использованию шифра RSA.
      \item Реализовать изученные алгоритмы в системе "Mathematica":
        \begin{enumerate}
          \item Тесты чисел на простоту.
          \item Генерация больших простых чисел.
          \item Алгоритмы факторизации.
          \item Алгоритмы шифрования и дешифрования.
          \item LLL-алгоритм и теорема Копперсмита.
          \item Криптоатаки на шифр RSA.
        \end{enumerate}
    \end{enumerate}