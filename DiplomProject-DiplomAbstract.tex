% !TEX root = DiplomProject.tex

\newpage
\chapter*{РЕФЕРАТ}
\addcontentsline{toc}{chapter}{Реферат}

КРИПТОСИСТЕМА, RSA, ПРОСТОЕ ЧИСЛО, ФАКТОРИЗАЦИЯ, КРИПТОАНАЛИЗ, ТЕСТ НА ПРОСТОТУ, ТЕОРЕМА КОППЕРСМИТА

\paragraph{} Дипломный проект представлен в виде пояснительной записки объемом n страниц, содержит 3 таблицы, 10 источников, 5 приложений.

\paragraph{} Данный дипломный проект является обобщением основных знаний и математических методов, связанных с криптосистемой RSA, работа также
	включает в себя реализацию основных алгоритмов, описанных в данной работе, в среде "Mathematica"

	Во введении содержится информации о криптосистемах с открытым ключом, даются основные определения, связанные с рассматриваемыми вопросами,
	осуществляется введение в криптосистему RSA, а также выполняется постановка задачи.

	В главе 1 "Построение простых чисел" рассмотрена проблема построения простых чисел, описаны алгоритмы, позволяющие строить простые числа, 
	кроме того уделено вниманием тестам чисел на простоту - тест на свидетельство простоты, тест Миллер-Рабина, сделано сравнение описанных алгоритмов
	с некоторыми другими - тест Соловея-Штрассена.

	В главе 2 "Проблема факторизации простых чисел" описаны основные подходы к факторизации чисел, рассмотрены и разобраны основные алгоритмы факторизации и примеры их работы, указаны скорости работы этих алгоритмов, проведено сравнение алгоритмов, выделены наиболее эффективные алгоритмы факторизации, описаны результаты, достигнутые с помощью алгоритмов факторизации.

	В главе 3 "Криптосистема RSA" рассмотрены основные алгоритмы криптосистемы RSA - создание ключей, шифрование и дешифрование, описаны основные
	подходы к криптоанализу RSA. Рассмотрены LLL-алгоритм, а также теорема Копперсмита, используемые при факторизации чисел при известной их 
	аппроксимации, описаны алгоритмы, использующие теорему Копперсмита для факторизации чисел.

\newpage
\chapter*{РЭФЕРАТ}
\addcontentsline{toc}{chapter}{Рэферат}

\newpage
\chapter*{ABSTRACT}
\addcontentsline{toc}{chapter}{Abstract}