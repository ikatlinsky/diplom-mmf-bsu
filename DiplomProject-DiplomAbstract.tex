% !TEX root = DiplomProject.tex

\newpage
\chapter*{РЕФЕРАТ}
\addcontentsline{toc}{chapter}{Реферат}

КРИПТОСИСТЕМА, RSA, ПРОСТОЕ ЧИСЛО, ФАКТОРИЗАЦИЯ, КРИПТОАНАЛИЗ, ТЕСТ НА ПРОСТОТУ, ТЕОРЕМА КОППЕРСМИТА

\paragraph{} Дипломный проект представлен в виде пояснительной записки объемом 58 страниц, содержит 4 таблицы, 12 источников, 5 приложений.

\paragraph{} Цель работы: изучение алгоритмов построения и поиска больших простых чисел, алгоритмов факторизации и анализ шифра RSA.

\paragraph{} Данный дипломный проект является обобщением основных знаний и математических методов, связанных с криптосистемой RSA, работа также
	включает в себя реализацию основных алгоритмов, описанных в данной работе, в среде <<Mathematica>>.	

	Во введении содержится информации о криптосистемах с открытым ключом, даются основные определения, связанные с рассматриваемыми вопросами,
	осуществляется введение в криптосистему RSA, а также выполняется постановка задачи.

	В главе 1 <<Построение простых чисел>> рассмотрена проблема построения простых чисел, описаны алгоритмы, позволяющие строить простые числа, 
	кроме того уделено внимание тестам чисел на простоту - тест на свидетельство простоты, тест Миллера-Рабина, сделано сравнение описанных алгоритмов. В рамках 
	главы реализованы алгоритмы Миллера-Рабина и Диемитко.

	В главе 2 <<Проблема факторизации целых чисел>> описаны основные подходы к факторизации чисел, рассмотрены и разобраны основные алгоритмы факторизации и примеры их работы, указаны скорости работы этих алгоритмов, проведено сравнение алгоритмов, выделены наиболее эффективные алгоритмы факторизации, описаны результаты, достигнутые с помощью алгоритмов факторизации. В рамках главы реализованы некоторые из рассмотренных алгоритмов факторизации.

	В главе 3 <<Криптосистема RSA>> рассмотрены основные алгоритмы криптосистемы RSA - создание ключей, шифрование и дешифрование, описаны основные
	подходы к криптоанализу RSA. Рассмотрены LLL-алгоритм, а также теорема Копперсмита, используемые при факторизации чисел при известной их 
	аппроксимации, описаны алгоритмы, использующие теорему Копперсмита для факторизации чисел. В рамках главы реализованы криптоатаки на шифр RSA, LLL-алгоритм, теорема 
	Копперсмита, примеры атак с использование теоремы Копперсмита.

\newpage

КРЫПТАСИСТЭМА, RSA, ПРОСТЫ ЛIК, ФАКТАРЫЗАЦЫЯ, КРЫПТААНАЛIЗ, ТЕСТ НА ПРАСТАТУ, ТЭАРЭМА КАППЕРСМИТА

\paragraph{} Дыпломны праект прадстаўлены ў выглядзе тлумачальнай запіскі аб'ёмам 58 старонак, змяшчае 4 табліцы, 12 крыніц, 5 прыкладанняў.

\paragraph{} Мэта працы: вывучэнне алгарытмаў пабудовы і пошуку вялікіх простых лікаў, алгарытмаў фактарызацыі і аналіз шыфра RSA.

\paragraph{} Дадзены дыпломны праект з'яўляецца абагульненнем асноўных ведаў і матэматычных метадаў, звязаных з крыптасістэмы RSA, праца таксама
ўключае ў сябе рэалізацыю асноўных алгарытмаў, апісаных у дадзенай працы, у асяроддзі <<Mathematica>>.

Ва ўводзінах ўтрымліваецца інфармацыі аб крыптасістэмы з адкрытым ключом, даюцца асноўныя вызначэння, звязаныя з разгляданымі пытаннямі,
ажыццяўляецца ўвядзенне ў крыптасістэму RSA, а таксама выконваецца пастаноўка задачы.

У главе 1 <<Пабудова простых лікаў>> разгледжана праблема пабудовы простых лікаў, апісаны алгарытмы, якія дазваляюць будаваць простыя лікі,
акрамя таго нададзена ўвага тэстам лікаў на прастату - тэст на сведчанне прастаты, тэст Мілера - Рабіна, зроблена параўнанне апісаных алгарытмаў. У рамках
працы рэалізаваны алгарытмы Мілера - Рабіна і Діемитка.

У главе 2 <<Праблема фактарызацыі лікаў>> апісаны асноўныя падыходы да факторизации лікаў, разгледжаны і разабраны асноўныя алгарытмы фактаризацыі і прыклады іх выкарыстання, пазначаны хуткасці працы гэтых алгарытмаў, праведзена параўнанне алгарытмаў, вылучаныя найбольш эфектыўныя алгарытмы фактыризацыі, апісаны вынікі, дасягнутыя з дапамогай алгарытмаў фактаризацыі. У рамках працы рэалізаваны некаторыя з разгледжаных алгарытмаў фактарызацыі.

У главе 3 <<Крыптасістэма RSA>> разгледжаны асноўныя алгарытмы крыптасістэмы RSA - стварэнне ключоў, шыфраванне і дэшыфраванне, апісаны асноўныя
падыходы да крыптааналізу RSA. Разгледжаны LLL-алгарытм, а таксама тэарэма Капперсміта, якія выкарыстоўваюцца пры фактарызацыі лікаў пры вядомай іх
апраксімацыі, апісаны алгарытмы, якія выкарыстоўваюць тэарэму Капперсміта для фактарызацыі лікаў. У рамках працы рэалізаваны крыптаатакі на шыфр RSA, LLL-алгарытм, тэарэма Капперсміта, прыклады нападаў з выкарыстаннем тэарэмы Капперсміта.

\newpage

CRYPTOSYSTEMS, RSA, PRIME NUMBER, FACTORIZATION, CRYPTO\-ANALYSIS, SIMPLICITY TEST, COPPERSMITH THEOREM

\paragraph{} The diploma project is presented in the form of an explanatory note of 58 pages, contains 4 tables, 12 references, 5 applications.

\paragraph{} Purpose: study of algorithms for constructing and searching large primes, factori\-zation algorithms and analysis cipher RSA.

\paragraph{} This thesis project is a generalization of the basic knowledge and mathematical methods associated with the cryptosystem RSA, work is also
includes an implemen\-tation of the basic algorithms described in this paper, in the application <<Mathema\-tica>>.

The introduction contains information about public-key cryptosystems, the basic definitions related to the issues at hand,
provided an introduction to the cryptosystem RSA, as well as the formulation of the problem is performed.

In Chapter 1 <<Building primes>> the problem of constructing primes described algorithms to build primes
also paid attention to Primality tests - test for evidence of simplicity, the Miller-Rabin test, made ​​a comparison algorithms described. Within
chapter Miller-Rabin and Diemitko algorithms were implemented.

In Chapter 2 <<The Factorization Problem>> primes basic approaches to the factorization properties, examined and dismantled the main factorization algorithms and examples of their work are the speed of these algorithms, the comparison algorithms, highlights the most efficient algorithms for factorization, described the results achieved with using factorization algorithms. Under Chapter implemented some of the considered factorization algorithms.

In Chapter 3 <<Cryptosystem RSA>> the basic algorithms cryptosystem RSA - key generation, encryption and decryption, describes the main
approaches to cryptanalysis RSA. LLL-algorithm is considered, as well as Theorem Coppersmith used in the factorization of numbers in their famous
approximation algorithms are described using Theorem Coppersmith for factoring numbers. Under Chapter implemented attacks on cipher RSA, LLL-algorithm theorem
Coppersmith, exampl\-es of attacks using Coppersmith's theorem.