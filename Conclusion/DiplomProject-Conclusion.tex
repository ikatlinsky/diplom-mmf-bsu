% !TEX root = ../DiplomProject.tex

\newpage
\chapter*{Заключение}
\addcontentsline{toc}{chapter}{Заключение}

	\paragraph{} В результате дипломной работы были выполнены следующее поставленные задачи:

	\begin{enumerate}
      \item Изучены способы построения больших простых чисел - рассмотрены алгоритмы построения простых чисел:
      	\begin{enumerate}
      		\item Тест Миллер-Рабина.
      		\item Алгоритм Диемитко.
      	\end{enumerate}
      \item Изучены и проведено сравнение некоторых алгоритмов факторизации - рассмотрены примеры экспоненциальных и субэкспоненциальных алгоритмов:
      	\begin{enumerate}
      		\item Экспоненциальные:
      			\begin{enumerate}
		      		\item Перебор возможных делителей.
		      		\item Метод факторизации Ферма.
		      		\item {$\rho$}-алгоритм Полларда.
		      	\end{enumerate}
      		\item Субэкспоненциальные:
      			\begin{enumerate}
		      		\item Алгоритм Диксона.
		      		\item Метод непрерывных дробей.
		      		\item Метод квадратичного решета.
		      		\item Метод решета числового поля.
		      	\end{enumerate}
      	\end{enumerate}
      \item Проведен криптоанализ шифра RSA - приведены примеры атак на шифр, а так же указаны уязвимые места при создании ключей:
      	\begin{enumerate}
      		\item Атаки, использующие недостатки при построении ключей и использовании шифра.
      		\item Атаки, основанные на теореме Копперсмита и LLL-алгоритме.    
      	\end{enumerate}
      \item Приведен пример практического использования шифра RSA - рассмотрено применение RSA на примере RSA-OAEP.      
      \item Реализованы изученные алгоритмы в системе "Mathematica":
      	\begin{enumerate}
          \item Тесты чисел на простоту.
          \item Генерация больших простых чисел.
          \item Алгоритмы факторизации.
          \item Алгоритмы шифрования и дешифрования.
          \item LLL-алгоритм и теорема Копперсита.
          \item Криптоатаки на шифр RSA.
        \end{enumerate}
\end{enumerate}

\section*{Построение больших простых чисел}
\addcontentsline{toc}{section}{Построение больших простых чисел}
	Были разобраны 2 различных алгоритма построения и получения больших простых чисел:
		\begin{itemize}
			\item Тест Миллера-Рабина - тест чисел на простоту 
			\item Теорема Диемитко - построение простых чисел        
		\end{itemize}

	Оба алгоритма имеют свои преимущества и недостатки, они выполняют одну и ту же задачу, но достигают ее разными способами, сравнение методов 
	приведено в таблице ниже

	\begin{table}[h]
		\centering	
		\begin{tabular}{|c|c|c|c|}	
			\hline  		     
		    \textbf{Метод} & \textbf{Вероятностный} & \textbf{Точный} & \textbf{Полиномиальный} \\ \hline		    
		    Миллер-Рабин        & \textbf{+}           & \textbf{-}                & \textbf{+}             \\ \hline
		    Диемитко            & \textbf{-}           & \textbf{+}                & \textbf{+}             \\	\hline	  		    
		\end{tabular} 
	\end{table}

	Алгоритм, основанный на теореме Диемитко, больше подходит для криптосистемы RSA, поскольку тест Миллера-Рабина использует неудобный подход, так как 
	перебор всех нечетный чисел в процессе поиска простого числа может занять длительное время, кроме того, результат выполнения алгоритма Миллера-Рабина вероятностный, то есть мы не можем дать гарантий того, что найденное число является простым, по крайней мере пока не доказана гипотеза Римана. В то же время, не смотря на то, что алгоритм Диемитко содержит вероятностную составляющую, результат алгоритма дает заведомо верный результат. Помимо этого, для получения большого простого числа нам необходимо небольшое простое число, выполнив несколько итераций алгоритма мы получим числу нужной нам размерности и нам не придется перебирать нечетный числа, как это было бы в тесте Миллера-Рабина.

	В итоге получается, что алгоритм Диемитко подход для криптосистемы RSA лучше, нежели тест Миллер-Рабина поскольку:
		\begin{itemize}
			\item Строит большие простые числа, а не тестирует нечетные числаТест Миллера-Рабина - тест чисел на простоту.
			\item Полиномиальный.
			\item Всегда дает точный результат, а не вероятностный.
		\end{itemize}

\section*{Факторизация больших целых чисел}
\addcontentsline{toc}{section}{Факторизация больших целых чисел}

	\paragraph{} При выполнении работы были рассмотрены различные алгоритмы факторизации больших целых чисел, рассматривались алгоритмы двух типов: 
	экспоненциальные и субэкспоненциальные, среди них были следующие:
		\begin{enumerate}
			\item Перебор возможных делителей.
			\item Метод факторизации Ферма.
			\item {$\rho$}-алгоритм Полларда.
			\item Алгоритм Диксона.
			\item Метод квадратичного решета.
			\item Метод решета числового поля.
		\end{enumerate}

	Среди перечисленных, наиболее эффективным на сегодняшний день является алгоритм решета числового поля, именно ему принадлежать все последние рекорды факторизации больших целых чисел, использование этого алгоритма вместо алгоритма квадратичного решета дало прирост практически в 10 раз.
	К примеру число из 130 знаков было разложено алгоритмом квадратичного решета за 5000 MIPS-лет, а то время как алгоритму решета числового поля потребовалось лишь 500 MIPS-лет.

	Однако, не смотря на такие серьезные результаты, выбор подходящего алгоритма факторизации является не тривиальной задачей и требует серьезных исследований, поскольку эффективность алгоритмов факторизации зависит от многих параметров. В частности таковыми параметрами являются применительно к алгоритму являются его тип - экспоненциальный или субэкспоненциальный, его вычислительная сложность. Кроме этих параметров существует наиболее важный - структура числа, подлежащего разложению. Получается что при выборе алгоритма необходимо руководствоваться не только свойствами алгоритма, но и подбирать алгоритм исходя из структуры числа. 

	Кроме того не менее важным является размерность числа - число битов и знаков числа, выбрав неподходящий алгоритм можно либо не получить результат, поскольку этот алгоритм не способен разложить данное число или наоборот, можно выбрать алгоритм, который разложит число, но ему для этого понадобятся большие вычислительные мощности. К примеру, не стоит использовать алгоритм решета числового поля для не очень больших числе, поскольку в этом алгоритме выполняется большое число промежуточных операций, таких как построение факторной базы, просеивание и так далее. В таком случае действие алгоритма невилируется и он станет не эффективным.

	Из этого можно сделать вывод: не смотря на большое число существующих алгоритмов факторизации, задача выбора эффективного алгоритма является не тривиальной, нужно проводить дополнительные исследования, чтобы исключить неоправданное использование вычислительных ресурсови выполнить задачу за наиболее короткий промежуток времени.

\section*{Криптоанализ шифра RSA}
\addcontentsline{toc}{section}{Криптоанализ шифра RSA}

	\paragraph{} В процессе работы были исследованы характеристики шифра RSA, а также приведен ряд атак на шифр. Атаки, приведенные в работе были двух типов:
		\begin{enumerate}
			\item Атаки, использующие уязвимости шифра, а именно ключей шифра RSA.
			\item Атаки, использующие теорему Копперсмита, то есть атаки при известных аппроксимациях чисел.
		\end{enumerate}

	Среди уязвимостей при создании ключей для криптосистемы RSA были рассмотрены следующие:
		\begin{enumerate}
			\item Малая экспонента $e$.
			\item Перебор возможных открытых текстов.
			\item Использование общих модулей.
			\item Мультипликативные свойства.
		\end{enumerate}

	Для исследования атак по аппроксимации была изучена теорема Копперсмита и LLL-алгоритм, который используется в алгоритме Копперсмита для нахождения малых корней полиномов по заданному модулю. Среди атак, использующих известные аппроксимации были изучены
		\begin{enumerate}
			\item Упрощенная проблема RSA.
			\item Атака Франклина-Райтера.
			\item Расширенная атака Хастаадта.
			\item Факторизация при известной аппроксимации.
			\item Атака Винера.
		\end{enumerate}

	\paragraph{} В результате были выделены следующие характеристики шифра RSA:

		\begin{enumerate}
			\item Распространенная криптосистема.
			\item Криптостойкая, при выполнении определенных условий.
			\item Легко реализуется.
			\item Алгоритмы создания ключей, шифрования и дешифрования работаю за полиномиальное время.
		\end{enumerate}	

	Кроме того, в настоящее время вычислительные мощности растут не так быстро, чтобы представлять серьезную угрозу хорошо построенному шифру, в связи с этим предлагаются следующие оптимизации для создания более надежного шифра RSA:
	
		\begin{enumerate}
			\item Экспонента $e$ должна быть достаточно большим числом.
			\item Минимальный размер числа $n$ - 768 бит, рекомендуется от 1024 до 2048.
			\item Простые числа $p$ и $q$ : $N=pq$ должны быть одинаковой длины, разность чисел $p - q$ не должна быть маленькой.
			\item Не рекомендуется использование общих модулей $N$ разными абонентами.
			\item Не рекомендуется использовать одинаковые экспоненты $e$.
			\item Рекомендуется использовать схемы дополнения, не не применять их в тривиальной форме.
		\end{enumerate}

	В конце работы была рассмотрена практическая реализация криптосистемы - RSA-OAEP, которая используется для шифрования при промышленном применении. В результате рассмотрения примера шфира \textit{RSA-OAEP} можно сделать выводы, что:
		\begin{enumerate}
			\item Использование шифра RSA в изначальном виде не оправдано.
			\item Для достижения необходимого уровня безопасности шифра необходимо использовать схемы дополнения.
			\item При использовании схем дополнения нужно учитывать возможности атак Франклина-Райтера и Хастадта.
		\end{enumerate}