\newpage
\chapter{Заключение}

\section{Выводы}

\newpage
\begin{thebibliography}{@@@@}	
	\bibitem{mah06}
		Маховенко, Е.Б. Теоретико-числовые методы в криптографии: учебное пособие / Е.Б. Маховенко. — Москва: Гелиос АРВ, 2006, — 320 с.
	\bibitem{tilb06}
		Тилборг, Х.К.А. ван. Основы криптологии. Профессиональное руководство и интерактивный учебник. / Х.К.А. ван Тилборг. — Моска: «Мир», 2006,
		— 471 с.
	\bibitem{vas03}
		Василенко О.Н. Теоретико-числовые алгоритмы в криптографии / О.Н. Василенко. – Москва: МЦНМО, 2003. – 328 с.
	\bibitem{ish11}
		Ишмухаметов Ш.Т. Методы факторизации натуральных чисел: учебное пособие / Ш.Т. Ишмухаметов.– Казань: Казан. ун. 2011.– 190 с.
	\bibitem{ngu10}
		Nguyen, P.Q. The LLL Algorithm / Phong Q. Nguyen, Brigitte Vallée. Springer, 2010.
	\bibitem{hoff08}
		Hoffstein, Jeffley An introduction to mathematical cryptography / J.Hoffstein, J. Pipher, J. H. Silverman. Springer 2008.
	\bibitem{chong12}
		Chong Hee Kim, Jean-Jacques Quisquater. Fault Attacks Against RSA-CRT Implementation / Fault Analysis in Cryptography – 2012, pp 125-136.
	\bibitem{nitaj13}
		Abderrahmane Nitaj. A new attack on RSA with two or three decryption exponents / Journal of Applied Mathematics and Computing. –
		2013, Volume 42, Issue 1-2, pp 309-319.	
	\end{thebibliography}