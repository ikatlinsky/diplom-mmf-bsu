% !TEX root = ../DiplomProject.tex

\newpage
\chapter*{Заключение}
\addcontentsline{toc}{chapter}{Заключение}

\begin{block}{Цель}
  Построение больших простых чисел и исследование криптостойкости шифра RSA.  
\end{block}  

\begin{block}{Задача 1}
  Разобрать способы построения простых чисел. 
\end{block} 

\begin{block}{Задача 2}
  Провести сравнение существующих алгоритмов факторизации.
\end{block}

\begin{block}{Задача 3}
  Провести криптоанализ шифра RSA.
\end{block}

\begin{block}{Задача 4}
  Реализовать изученные алгоритмы в пакете Mathematica.
\end{block}

\begin{center}
	\begin{enumerate}
		\item Разобраны, сравнены и реализованы алгоритмы получения простых чисел - тест Миллер-Рабина, теорема Диемитко. + результаты
		\item Разобраны, сравнены и реализованы алгоритмы факторизации - экспоненциальные и субэкспоненциальные. + результаты
		\item Проведен криптоанализ шифра RSA, указаны слабые места при создании ключей для шифра. + результаты
		\item Реализованы криптоатаки на шифр RSA. + перечислить что реализовано
	\end{enumerate}

	\begin{block}{Реализовано}
		\begin{itemize}
	        \item Тесты чисел на простоту
	        \item Генерация больших простых чисел
	        \item Алгоритмы факторизации
	        \item Алгоритмы шифрования и дешифрования
	        \item LLL-алгоритм + теорема Копперсита
	        \item Криптоатаки на шифр RSA
		\end{itemize}
	\end{block}	
\end{center}

\begin{center}		

	\begin{block}{RSA}
		\begin{enumerate}
			\item Распространенная криптосистема
			\item Криптостойкая, при выполнении определенных условий
			\item Легко реализуется
			\item Алгоритмы создания ключей, шифрования и дешифрования работаю за полиномиальное время
		\end{enumerate}	
	\end{block}

	\begin{block}{Рекомендации при использовании шифра RSA}
		\begin{enumerate}
			\item Экспонента $e$ должна быть достаточно большим числом
			\item Минимальный размер числа $n$ - 768 бит, рекомендуется от 1024 до 2048
			\item Простые числа $p$ и $q$ : $N=pq$ должны быть одинаковой длины, разность чисел $p - q$ не должна быть маленькой
			\item Не рекомендуется использование общих модулей $N$ разными абонентами
			\item Не рекомендуется использовать одинаковые экспоненты $e$
		\end{enumerate}	
	\end{block}

\end{center}