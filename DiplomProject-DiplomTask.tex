% !TEX root = DiplomProject.tex

\begin{titlepage}

	\begin{center}
		\textbf{БЕЛОРУССКИЙ ГОСУДАРСТВЕННЫЙ УНИВЕРСИТЕТ} \\
		\vspace{0.5cm}		
		Механико-математический факультет \\*		
		кафедра дифференциальных уравнений и системного анализа \\
	\end{center}

	\begin{flushright}
		Заведующий кафедрой, профессор \\		
		\rule{2.5cm}{0.1pt} В.И.Громак \\
		<<\rule{0.75cm}{0.1pt}>> \rule{2.5cm}{0.1pt} 2014 г.	
	\end{flushright}

	\vspace{0.5cm}

	\begin{center}
		\textbf{Задание} \\
		\textbf{на выполнение дипломной работы}
	\end{center}

	\textbf{Студенту \rule{12cm}{0.1pt}}

	\begin{enumerate}	
		\item \textbf{Тема дипломной работы:} Анализ криптосистемы RSA.
		\item \textbf{Срок сдачи студентом выполненной дипломной работы:}  май 2014.
		\item \textbf{Краткое обоснование актуальности предлагаемой работы с указанием основных источников информации.} \\
			Самой известной и применимой на практике криптосистемой с открытым ключом является криптосистема RSA. Поэтому важным является изучение криптостойкости данной системы. Данный шифр является одним из первых полученных ассиметричных криптосистем и существует достаточно много публикаций о данном шифре. Автору предлагается систематизировать обилие информации о криптосистеме RSA и реализовать изученные алгоритмы в пакете Mathematica. \\

			Литература. \\
				1. Маховенко, Е.Б. Теоретико-числовые методы в криптографии: учебное пособие /  Е.Б. Маховенко. — Москва: Гелиос АРВ, 2006, — 320 с. \\
				2. Тилборг, Х.К.А. ван. Основы криптологии. Профессиональное руководство и интерактивный учебник. /  Х.К.А. ван Тилборг. — Моска: «Мир», 2006, — 471 с. \\
				3. Василенко О.Н. Теоретико-числовые алгоритмы в криптографии / О.Н. Василенко. – Москва: МЦНМО, 2003. – 328 с. \\
				4. Nguyen, P.Q. The LLL Algorithm / Phong Q. Nguyen, Brigitte Vallée. Springer, 2010. \\
				5. Hoffstein, Jeffley An introduction to mathematical cryptography / J. Hoffstein, J. Pipher, J. H. Silverman. Springer 2008.  \\
				6. Chong Hee Kim, Jean-Jacques Quisquater. Fault Attacks Against RSA-CRT Implementation // Fault Analysis in Cryptography – 2012, pp 125-136. \\
				7. Abderrahmane Nitaj. A new attack on RSA with two or three decryption exponents // Journal of Applied Mathematics and Computing. – 2013, Volume 42, Issue 1-2, pp 309-319.			
		\item \textbf{Содержание дипломной работы.} \\
			1) Генерация больших простых чисел. \\
			2) Факторизация натуральных чисел. \\
			3) Криптосистема RSA. \\
			4) Криптоанализ RSA. \\
			5) Практическое применение RSA. 
		\item \textbf{Руководитель дипломной работы} Чергинец Дмитрий Николаевич, доцент кафедры дифференциальных уравнений и системного анализа, кандидат физ.-мат. наук.
	\end{enumerate}

	\vspace{0.5cm}

	\begin{flushleft}
		Руководитель дипломной работы \rule{7cm}{0.1pt} \\

		\vspace{0.25cm}

		Задание к исполнению принял \\
		Студент \rule{7cm}{0.1pt} \\
		<<\rule{0,75cm}{0.1pt}>> \rule{4cm}{0.1pt} 2014 г.
	\end{flushleft}

\end{titlepage}