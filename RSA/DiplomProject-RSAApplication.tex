% !TEX root = ../DiplomProject.tex

\section{Применение RSA}

\paragraph{} Крайне не рекомендуется использовать \textit{RSA} в его наиболее простой форме, существуют требования, выдвигаемые к асимметричным криптосистемам
- \cite[страницы 329-331]{may10}.

  Современная асимметричная криптосистема может считаться стойкой, если злоумышленник, имея два открытых текста {$M_1$} и 
  {$M_2$}, а также один шифротекст {$C_b$} не может с вероятностью большей, чем 0.5 определить какому из двух 
  открытых текстов соответствует шифротекст {$C_b$}. Приведем несколько примеров подобной ситуации.

  Предположим злоумышленник следит за сообщениями объектов \textit{А} и \textit{B}. Пусть 
  \textit{B} в открытом виде отправил \textit{A} сообщение, а \textit{A} односложно отвечает \textit{B} на это сообщение, шифруя свой ответ открытым ключом
  \textit{B}. Злоумышленник перехватывает шифротекст и предполагает, что в нем зашифровано либо Да, либо Нет, чтобы узнать ответ \textit{A} он шифтрует открытым ключом \textit{B} слово Да и если полученный криптотекст совпадет 
  с перехваченным, то это означает, что \textit{A} ответил Да, в противном же случае злоумышленник поймет, что ответом было Нет.

  Из примера видно, что \textit{RSA} не так надежна как требуется, поскольку злоумышленнику в данном случае не составит труда взломать шифр. Чтобы избежать подобных ситуаций, достаточно чтобы алгоритм добавлял к тексту некоторую случайную информацию, которую бы невозможно было предугадать. 

  Рассмотрим следующий пример: пусть злоумышленник получил доступ к алгоритму дешифрования. Таким образом любой он может расшифровать любой шифротекст. Далее злоумышленник создает два открытых текста {$M_1$} и {$M_2$}. 
  Один из этих текстов шифруется и полученный в результате шифротекст {$C_b$} возвращается злоумышленнику. Задача злоумышленника 
  угадать с вероятностью большей чем 0.5 какому из сообщений {$M_1$} и {$M_2$} соответствует шифротекст {$C_b$}. Говорят что криптосистема стойкая, если злоумышленник не сможет указать какому исходному тексту соответствует {$C_b$} с вероятностью большей 0.5.

  Рассмотрим насколько криптостойкой окажется \textit{RSA} в данном случае. Злоумышленник имеет два сообщения {$M_1$} и 
  {$M_2$}. А также шифротекст {$C_{b}=M^{e}_{1}(mod \: n)$}. Зная открытый ключ \textit{e}, он может 
  создать сообщение {$C'=2^{e} C_{b}(mod \: n)$}. Далее он расшифровывает сообщение $C'$ и получает:

    \begin{equation}
      \textit{{$M'=C'^d(mod \: n)=2^{e d}M_{1}^{e d}(mod \: n)=2M_{1}(mod \: n).$}} 
    \end{equation}

  Таким образом вычислив {$\rfrac{M'}{2}$} злоумышленник узнает {$M_1$}. Этот результат говорит в неприемлемости использования \textit{RSA} в его изначальном виде на практике.

\subsection{RSA-OAEP}

\paragraph{} Таким образом, можно заметить, что \textit{RSA} в изначальном виде при промышленном применении (\textit{PGP},\textit{SSL}) не приемлемо. Алгоритм должен сперва добавлять к этим данным блоки содержащие случайный набор бит, то есть использовать схемы дополнения. И только после этого полученный
  результат шифруется. Таким образом вместо изначальной схемы {$c=m^{e}(mod \: n)$} получаем пригодную в промышленном применении схему 
  {$c=(m||r)^{e}(mod \: n)$}, где \textit{r} - случайное число. В настоящее время использование
  \textit{RSA} без схем дополнения является нарушением стандартов.

  В настоящее время к схемам дополнения также предъявляются требования о том, чтобы дополнительные блоки помогали определить был ли шифротекст получен в результате работы 
  шифрующей функции. В случае, если будет обнаружено, что шифротекст смоделирован злоумышленником, вместо расшифрованных данных 
  злоумышленник получит сообщение о несоответствие данных реальному шифротексту.

  В \textit{RSA} при подписи и при шифровании данных используют две различные схемы дополнений. Схема, реализуемая для подписи документов, называется 
  \textit{RSA-PSS(probabilistic signature scheme)} или вероятностная схема подписи. Схема, используемая при шифровании - 
  \textit{RSA-OAEP(Optimal asymmetric encryption padding)} или оптимизированное асимметричное дополнение шифрования, на примере \textit{OAEP} рассмотрим 
  шифрование и дешифрование сообщений в \textit{RSA}.

  И так чтобы зашифровать абсолютно любое сообщение в \textit{RSA-OAEP} делается следующее:
    \begin{enumerate}
	    \item Выбираются две хеш-функции {$G(x)$} и {$H(x)$} таким образом, чтобы суммарная длина результатов хеш-функций не превышала длины 
	      ключа \textit{RSA}.
	    \item Генерируется случайная строка битов {$l$}. Длина строки должна быть равна длине результата хеш-функции {$H(x)$}.
	    \item Сообщение {$M$} разбивают на блоки по \textit{{$k$}-бит}. Затем к каждому полученному блоку {$m$} дописывают {$(n-k)$} нулей. 
	      Где \textit{{$n$}-длина} хеш-функции {$G(x)$}.
	    \item Определяют следующий набор бит: {$ \{m||0^{(n-k)} \oplus G(l)\}||\{l \oplus H(m||0^{(n-k)} \oplus G(l))\} $}
	    \item Полученные биты представляют в виде целого числа {$M_1$}
	    \item Криптотекст получают по формуле: {$ C=M_{1}^{e}(mod \: n) $}
    \end{enumerate}

  Процесс дешифрования выглядит следующим образом:
    \begin{enumerate}
	    \item Находят {$M_1$} по формуле \textit{$ M_{1}=C^{d}(mod \: n) $}
	    \item Переводят число $M_1$ в набор бит и отсекают левую часть длиной $n$, где {$n$}-длина 
	      хеш-функции {$G(x)$}. Обозначим эти биты {$T$}. {$T= \{m||0^{(n-k)} \oplus G(l)\} $}. Все 
	      остальные биты являются правой частью.
	    \item Находим {$ H(T)=H(m||0^{(n-k)} G(l)) $}
	    \item Зная {$H(T)$} получаем {$l$}, поскольку знаем {$l \oplus H(T)$} - это правая часть блока
	    \item Вычислив {$l$}, находим {$m$} из {$T \oplus G(l)$}, поскольку {$ T=\{m||0^{(n-k)} \oplus G(l)\} $}
	    \item Если {$m$} заканчивается {$(n-k)$}-нулями значит сообщение зашифровано правильно. Если нет то это значит, что шифротекст 
	      некорректен, а следовательно он скорее всего подделан злоумышленником.
    \end{enumerate}

  Таким образом \textit{RSA} это не только возведение в степень по модулю большого числа. Это еще и добавление избыточных данных позволяющих реализовать 
  дополнительную защиту вашей информации.

  \subsection{Выводы}

  \paragraph{} При рассмотрении примеров использования криптосистемы RSA в изначальном виде и описании возможных проблем, а также при изучении схемы \textit{RSA-OAEP} можно сделать следующие выводы:
    \begin{enumerate}
      \item Не рекомендуется использовать схему RSA в изначальном виде.
      \item При промышленном применении шифра RSA следут использовать схемы дополнения.
      \item К схемам дополнения предъявляются определенные требования: это должен быть случайный набор бит, дополнительные блоки должны указывать, был ли шифротекст создан в результате действия алгоритма шифрования.
      \item При использовании схем дополнения не стоит забывать про атаки Хастадта и Франклина-Райтера.
    \end{enumerate}