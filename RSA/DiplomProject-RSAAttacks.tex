% !TEX root = ../DiplomProject.tex

\section{Атаки на RSA}

\subsection{Упрощенная проблема RSA}

\paragraph{} Проблемой RSA является задача обращения функции RSA. Имея {$m^e (mod \: N)$} необходимо найти уникальный {$e$}-ый корень {$m \in Z_N$}, 
  предполагается, что эту задачу сложно решить для произвольного {$m$}, однако проблема является тривиальной для малых {$m$} и {$e$}. А именно, если
  {$m < N^\frac{1}{e}$}, тогда {$m^e (mod \: N) = m^e$} над {$Z$}.
  
  \subsubsection{Проблема RSA}
  Дано: {$m^e (mod N)$} \\   
  Найти: {$m \in Z_N$}
  
\paragraph{} Копперсмит расширил этот результат на случай, когда {$m$} не является малым, но мы знаем {$m$} с некоторой точностью. Точнее, мы знаем аппроксимацию
  {$m'$} такую, что {$m = m' + x_0$} для неизвестного {$x_0$}: {$|x_0| \le N^\frac{1}{e}$}. Это утверждение можно представить в виде полинома
  
    \begin{equation}
     f(x) = (m' + x)^e - m^e (mod \: N).
    \end{equation}
    
  \subsubsection{Упрощенная проблема RSA}
  Дано: {$m^e, m', |m-m'| \le N^\frac{1}{e}$} \\   
  Найти: {$m \in Z_N$}  
  
\paragraph{} Описанный выше результат можно интерпретировать следующим образом: пусть {$e = 3$}, тогда злоумышленник, который знает 2/3 сообщения, может узнать оставшуюся
  1/3 часть сообщения {$m$} за полиномиальное время. Это может случиться при использовании шаблонных сообщений вида \textit{Сегодня пароль: xxxx}, часть 
  \textit{Сегодня пароль:} повторяется каждый раз, поэтому злоумышленник легко сможет вычислить пароль - \cite[страницы 328-329]{may10}.

\subsection{Атака Франклина-Райтера}

\paragraph{} Предположим, что у нас есть открытый ключ RSA, принадлежащий A. Атака \textit{Франклина-Рейтера} применяется в следующей ситуации. B хочет отправить A два шифрованных 
  сообщения $m_1$ и $m_2$, связанных друг с другом с помощью следующего открытого многочлена $ m_2 = f(m_1)(mod \: n) $
  
  По соответствующим шифротекстам $c_1$ и $c_2$ нападающий имеет хорошие шансы раскрыть сообщения $m_1$ и $m_2$ при любой небольшой
  шифрующей экспоненте $e$. Атака наиболее проста, если $f = aх + b $ и $e=3$, причем $a$ и $b$ фиксированы и известны атакующему. 
  Нападающий знает, что $m_2$ - корень по модулю $n$ многочленов
  
    \begin{equation}
      g_1(x) = x^3 - c_2, \: g_2(x) = (f(x))^3 - c_1
    \end{equation}

  Поэтому линейная функция $x - m_{2}$ делит как $ g_1(x) $, так и $ g_2(x) $.
  
  Можно показать, что в случае $f = aх + b$ и $e = 3$ искомый наибольший общий делитель (если таковой существует) должен быть линейным множителем 
  многочлена $x - m_2$, то есть совпадать с ним. Значит, нападающий найдет $m_2$, а затем $m_1$.
  
  Обобщение \textit{Копперсмита} атаки \textit{Франклина-Рейтера} строится по пути, на котором можно получить дополнительный результат из атаки Хастадта. Предположим, что сообщение $m$ 
  перед шифрованием пополняется некоторыми случайными данными. Например, если $n$ - \textit{n-битовый} модуль \textit{RSA}, а $m$ - \textit{k-битовое} сообщение, 
  то можно добавить $n-k$ случайных битов либо в начало, либо в конец сообщения. Пусть $m' = 2^{n-k}m + r $, где $r$ - зависящее от 
  $m$ случайное число длины $n-k$. Копперсмит показал, что такое пополнение сообщения нестойко.
  
  Предположим, что $B$ дважды посылает одно сообщение $A$, то есть возникает два шифротекста $ m_1 = 2^{n-k}m + r_1 $ и
  $ m_2 = 2^{n-k}m + r_2 $ соответствующих сообщениям где $r_1$ и $r_2$ - два разных случайных числа, состоящие из 
  $n - k$ битов. Атакующий вводит обозначение $ y_0 = r_2 - r_1 $ и пытается решить
  систему уравнений
  
    \begin{equation}
	  \begin{cases}
	    g_1(x,y) = x^e - c_1 \\
	    g_2(x,y) = (x+y)^e - c_2
	  \end{cases}    
      \end{equation}
      
  Он вычисляет результант $h(y)$ многочленов $g_1(x,y)$ и $g_2(x,y)$ относительно переменной $x$. Теперь $y_0$ -
  маленький корень многочлена $h(y)$ степени $e^2$. Опираясь на \textit{теорему Копперсмита} (\ref{th:commpersmith}), нападающий находит разность $r_2 - r_1$ и 
  узнает $m_2$, используя метод \textit{атаки Франклина-Рейтера} - \cite[страницы 331-332]{may10}.

\subsection{Расширенная атака Хастаадта}

\paragraph{} Предположим, что мы хотим транслировать простое RSA зашифрованное сообщение к группе из {$k$} получателей, которые имеют публичные
  экспоненты {$e$} и взаимно простые модули {$N_1, \dots, N_k$}. То есть, мы отправляем сообщения {$m^e (mod N_1), \dots, m^e (mod N_k)$}. 
  Исходя из этой информации, злоумышленник может вычислить {$m^e (mod \prod_{i=1}^{k}{N_i})$}. Если {$m^e$} меньше, чем произведение модулей, то он 
  может вычислить {$m$} по {$e$}-му корню над целыми числами. Если все {$N_i$} имеют одинаковые размерности, то необходимо {$k > e$} сообщений,
  чтобы вычислить {$m$}.
  
  Предполагая, что сообщения {$m$} меньше {$min_j \{ N_j \}$}, обрабатываем их используя известные полиномиальные отношения {$g_1, \dots, g_k$}
  со степенями $\delta_1, \dots,$ $\delta_k$ соответственно.
  
  Применяя \textit{теорему Копперсмита} (\ref{th:commpersmith}) с параметрами {$\beta,c=1$} мы можем найти все корни {$m$}, удовлетворяющие ограничению
  {$m \le N^\frac{1}{\delta} = \prod_{i=1}^{k}{N\textsuperscript{{$\frac{1}{\delta_i e_i}$}}_i}$} - \cite[страницы 331-333]{may10}.

\subsection{Факторизация при известной аппроксимации}

\paragraph{} Пусть {$N = pq$}, {$p > q$}. Предположим, что мы знаем наиболее важные биты числа {$p$}. Наша задача найти факторизацию числа
  {$N$} за полиномиальное время.
  
  В 1985, Риверст и Шамир опубликовали алгоритм, который факторизует {$N$} зная {$\frac{2}{3}$} битов числа {$p$}. Копперсмит улучшил этот результат
  до {$\frac{3}{5}$} в 1995. Годом позже он опубликовал алгоритм, который использовал {$\frac{1}{2}$} битов числа {$p$}.
  
  Проблему разложения чисел при знании битов одного из чисел разложения также можно свести к проблеме решения одномерного полиномиального
  уравнения, используя \textit{LLL-алгоритм}. Предположим нам известна половина битов числа {$p$}, тогда нам известна аппроксимация {$p'$}
  числа {$p$} такая, что {$|p-p'| \le N^\frac{1}{4}$}.
  
  \subsubsection{Проблема факторизации}
  Дано: {$N = p q$} \\   
  Найти: {$p$}  
  
  \subsubsection{Проблема факторизации при известной аппроксимации}
  Дано: {$ N = p q, p': |p-p'| \le N^\frac{1}{4} $} \\   
  Найти: {$p$}  
  
  Наша цель - найти недостающие биты числа {$p$}, то есть мы хотим найти корни одномерного, линейного полинома
  
    \begin{equation}
      f(x) = p' + x (mod \: p).
    \end{equation}

  Заметим, что {$p-p'$} является корнем {$f(x)$} при абсолютных значениях меньших, чем {$N^\frac{1}{4}$}.
  
  Применим \textit{теорему Копперсмита} (\ref{th:commpersmith}) с полиномом {$f(x) = p' + x$}, то есть с параметрами {$\delta = 1, \beta = \frac{1}{2}, c = 1$}.
  Кроме того, мы можем найти все корни {$x_0$} размерности
  
    \begin{equation}
     |x_0| \le N^\frac{\beta^2}{\delta} = N^\frac{1}{4}.
    \end{equation}
    
  Эти рассуждения и действия позволяют нам найти недостающие биты числа {$p$} за полиномиальное время с использованием \textit{LLL-алгоритма}
  и \textit{теоремы Копперсмита} (\ref{th:commpersmith}), а значит решить проблемы факторизации числа {$N$} - \cite[страницы 333-335]{may10}.
\subsection{Атака Винера}

\paragraph{} Мы уже отмечали, что в алгоритме \textit{RSA} для ускорения операций с открытым ключом используют малые шифрующие экспоненты. В некоторых же 
  приложениях этой криптосистемы существеннее ускорить процессы расшифровывания. Поэтому имеет смысл выбирать небольшую расшифровывающую экспоненту
  $d$. Ясно, что при этом получается большое значение открытой экспоненты $e$. Слишком маленькое число в качестве секретной экспоненты $d$ мы 
  брать не можем, поскольку атакующий определит ее простым перебором. Более того, учитывая атаку Винера, опирающуюся на непрерывные дроби,
  необходимо выбирать $d$ среди чисел, размер которых не меньше, чем \textit{{$ \frac{1}{3} N^\frac{1}{4} $}}
  
  По вещественному числу $ \alpha \in R $ определим последовательности:
  
  \begin{subequations}
      \begin{center}
	$\alpha_0 = \alpha, p_0 = q_0 = 1, p_1 = a_0 a_1 + 1, q_1 = a_1$,\\
	$a_i = \lfloor \alpha_i \rfloor, \alpha_{i+1} = \frac{1}{\alpha_i - a_i}$, \\
	$p_i = a_i p_{i-1} + p_{i-2}, q_i = a_i q_{i-1} + q_{i-2}, i \ge 2$,
      \end{center}
  \end{subequations}
  
  Целые числа $ \alpha_0, \alpha_1, \alpha_2, \dots $ называются непрерывной дробью, представляющей $\alpha$, а рациональные числа $ \frac{p_i}{q_i} $
  - подходящими дробями. Каждая из подходящих дробей несократима, а скорость роста их знаменателей сравнима с показательной.
  
  Одним из важных результатов теории непрерывных дробей является то, что если несократимая дробь удовлетворяет неравенству:
  
    \begin{equation}
	   | \alpha - \frac{p}{q} | \le \frac{1}{2 q^2},
    \end{equation}

  то $ \frac{p}{q} $ - одна из подходящих дробей в разложении $\alpha$ в непрерывную дробь.
  
  Винер предлагает использовать непрерывные дроби при атаке на \textit{RSA} следующим образом. Пусть у нас есть модуль $N = p q$, причем $q < р < 2q$. 
  Допустим, что наша расшифровывающая экспонента удовлетворяет неравенству $d < \frac{1}{3} N^\frac{1}{4} $, и нападающему это известно. Кроме того, 
  ему дана шифрующая экспонента $e$, обладающая свойством 
  
    \begin{equation}
      e d = 1 (mod \phi), \phi = \phi(N) = (p - 1)(q - 1)
    \end{equation}
  
  Будем также считать, что $e < \phi$, поскольку это выполнено в большинстве приложений. Заметим, из предположений следует существование такого целого 
  $k$, что $e d - k \phi = 1$. Следовательно,
  
    \begin{equation}
     |\frac{e}{\phi} - \frac{k}{d}| = \frac{1}{d \phi}
    \end{equation}
    
  Поскольку $\phi \approx n$, получаем, что
  
    \begin{equation}
     |n - \phi| = |p + q - 1| < 3 \sqrt{n}
    \end{equation}

  Отсюда можно сделать вывод, что $ \frac{e}{n} $ - довольно хорошее приближение в $ \frac{k}{d} $. Действительно,
  
    \begin{subequations}
      \begin{center}
	$|\frac{e}{n} - \frac{k}{d}| = |\frac{e d - n k}{d n}| = |\frac{e d - k \phi - n k + k \phi}{d n}| = $ \\
	$= |\frac{1 - k(n - \phi)}{d n}| \le |\frac{3 k \sqrt{n}}{d n}| = |\frac{3 k}{d \sqrt{n}}|$
      \end{center}
    \end{subequations}
    
  Поскольку $e < \phi$, очевидно, $k < d$. Кроме того, по предположению $d < \frac{1}{3} N^\frac{1}{4} $. Значит, 
  $ |\frac{e}{n} - \frac{k}{d}| < \frac{1}{2 d^2} $ - \cite[Глава 14, страницы 354-356]{smart05}.
  
  Поскольку $GCD(k, d) = 1$, мы видим, что $\rfrac{k}{d}$ - подходящая дробь в разложении дроби $\rfrac{e}{n}$ в непрерывную. Таким образом, раскладывая число $\rfrac{e}{n}$ в непрерывную дробь, 
  можно узнать 
  расшифровывающую экспоненту, поочередно подставляя знаменатели подходящих дробей в выражение $(M^{e})^{d} \equiv M (mod \: n)$ для некоторого 
  случайного числа $M$. 
  Получив равенство, найдем $d$. Общее число подходящих дробей, которое нам придется при этом проверить, оценивается как $О(ln n)$. 
  Таким образом, изложенный метод дает линейный по сложности алгоритм определения секретного ключа в системе \textit{RSA}, если последний не 
  превосходит $\frac{1}{3} N^\frac{1}{4} $ - \cite[Глава 9.4, страницы 174-177]{tilb06}.
  
  \begin{example}
    В качестве примера рассмотрим модуль RSA, равный $n = 9449868410449$. Пусть открытый ключ криптосистемы задан как $e = 6792605526025$,
    а секретный ключ удовлетворяет неравенству $d < \frac{1}{3} N^{\frac{1}{4}} \approx 584$. Разложим число $ \alpha = \frac{e}{n} $ непрерывную дробь и проверим 
    знаменатель каждой подходящей дроби: не является ли он секретным ключом. Подходящие дроби разложения $ \alpha $ имеют вид:
    
      \begin{equation}
	        1, \frac{2}{3}, \frac{3}{4}, \frac{5}{7}, \frac{18}{25}, \frac{23}{32}, \frac{409}{569}, \frac{1659}{2308}, \dots
      \end{equation}

    Поочередно проверяя знаменатели, убедимся, что $d = 569$, то есть знаменатель седьмой подходящей дроби - искомый секретный ключ.
  \end{example}
  
  
\subsection{Выводы}

\paragraph{} Изобретение \textit{LLL-алгоритма} стало базисом и отправной точкой для создания эффективного алгоритма и \textit{теоремы Копперсмита}, которые
  решают проблему поиска малых решений для полиномиальных уравнений. Это открыло совершенно новую линию исследований и новые направления решения
  сложных проблем, такие как проблема факторизации, позволила рассматривать вопросы под совершенно новым углом. По сравнению с
  традиционными методами, такими как \textit{метод эллиптических кривых}, \textit{метод решета числового поля}, \textit{LLL-алгоритм} работает за полиномиальное время,
  однако способен решать лишь приближенные задачи факторизации и отыскания секретного ключа для \textit{RSA}.
  
  Сегодня, \textit{упрощенные методы} все еще далеки от желаемого результата и эффективности и по некоторым позициям проигрывают традиционным методам,
  однако наблюдается устойчивый прогресс в поиске новых приложений, методов и вариаций алгоритмов, улучшаются результаты, все больше ограничений
  снимается. С исследовательской точки зрения еще довольно новый метод поиска решений, базирующийся на LLL-методе, представляет большой интерес,
  содержит в себе много результатов, которые будут открыты.

  После рассмотрения различных криптоатак на шифр RSA можно сделать следующие выводы:
    \begin{enumerate}
      \item Благодаря теореме Копперсмита можно найти решение полиномиального уравнения в кольце вычетов при условии, что решение мало. Алгоритм поиска таких решений основан на построении решетки, а именно на LLL-алгоритме.
      \item Частичное раскрытие ключа шифра может привести к полному раскрытию ключа и, как следствие, взлому шифра RSA. Это вытекает из теоремы Копперсмита и примеров ее использования.
      \item Не стоит использовать малую шифрующую экспоненту в шифре RSA, поскольку это может привести к взлому с помощью атаки Винера или китайской теоремы об остатках.
      \item Секретный ключ $d$ необходимо выбирать среди чисел не меньших, чем $\frac{1}{3} N^{\frac{1}{4}}$
      \item Необходимо осторожно пользоваться схемами дополнения в шифре RSA, поскольку это может привести к взлому с помощью атак Хастадта и 
      Франклина-Райтера. К примеру стоит избегать ситуаций, при которых происходит отправка двух сообщений, которые связаны друг с другом при помощи соотношения

        \begin{equation}
          m_2 = f(m_1)(mod \: N).
        \end{equation}
      Либо использовать дополнения, зависящие от определенных параметров, специфических для получателя
        \begin{equation}
          c_i=(i \: 2^H m)^e(mod \: N_i),
        \end{equation}
      поскольку это может привести к взлому путем решения уравнений или системы уравнений типа 
        \begin{equation}
          g_i(x) = (i \: 2^H x)^e(mod \: N_i) - c_i,
        \end{equation}
    \end{enumerate}
  
  Имеющиеся на сегодня результаты по криптоатакам на шифр RSA отображены в таблице \ref{table-prime-attacks}.

\begin{table}[ht]
    \centering
    \begin{tabular}{@{}p{2.5cm}p{2.5cm}cp{2.5cm}p{4.5cm}@{}}
    \toprule
    \textbf{Число знаков} & \textbf{Число битов} & \textbf{Дата решения} & \textbf{Число MIPS-лет} & \textbf{Алгоритм} \\ \midrule
    100                              & 332                  & Апрель 1991 г         & 7                       & Квадратичное решето              \\
    110                              & 365                  & Апрель 1992 г.        & 75                      & Квадратичное решето              \\
    120                              & 398                  & Июнь 1993 г.          & 830                     & Квадратичное решето              \\
    129                              & 428                  & Апрель 1994 г.        & 5000                    & Квадратичное решето              \\
    130                              & 431                  & Апрель 1996 г.        & 500                     & Решето в поле чисел общего вида  \\ 
    \bottomrule
    \end{tabular}
    \caption{Решение задач факторизации}
    \label{table-prime-attacks}
  \end{table}

Единицей меры сложности задачи в данном случае является \textit{MIPS-го}д - объем работы, выполняемой в течение года процессором, осуществляющим обработку одного миллиона команд в секунду, что примерно эквивалентно выполнению {$3\times10^{13}$} команд.  
  
Известны также более современные результаты факторизации, а именно:  
    \begin{itemize}
      \item Между январем и августом 1999 года с использованием общего метода решета числового поля было факторизовано RSA-155. Вычисления снова производились с 
	привлечением большого количества людей, а финальные вычисления - на суперкомпьютере C916.
      \item В апреле 2003 года Франке и другие объявили о факторизации RSA-160. При разложении использовалось около сотни CPU.
      \item В декабре 2003 Франке, Клеинджанг факторизовали 174-значное число, используя ресурсы BSI и Боннского университета.
      \item В мае 2005 176-значный сомножитель числа 11281 + 1 был найден Аоки, Кида, Шимоямой и Уедой.
    \end{itemize}
  
Угроза ключам большой длины здесь двойная: непрерывный рост вычислительной мощи современных компьютеров и непрерывное 
  усовершенствование алгоритмов разложения на множители.



