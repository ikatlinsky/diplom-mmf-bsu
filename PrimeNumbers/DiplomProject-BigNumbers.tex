\section{Построение больших простых чисел}

\paragraph{} Существует довольно эффективный способ убедиться, что заданное число является составным, не разлагая это число на множители. 
Согласно малой теореме Ферма, если число \textit{n} простое, то для любого целого \textit{a}, не делящегося на \textit{n}, выполняется сравнение
\begin{equation}
 \textit{a\textsuperscript{ n-1} {$\equiv$} 1 (mod n)}.
\end{equation}
Если же при каком-то \textit{a} это сравнение нарушается, можно утверждать, что \textit{n} - составное. Вопрос только в том, как найти для составного \textit{n}
целое число \textit{a}, не удовлетворяющее (***). Можно, например, пытаться найти необходимое число \textit{a}, испытывая все целые числа подряд, 
начиная с \textit{2}. Или попробовать выбирать эти числа случайным образом на отрезке \textit{1 < a < n}.

  К сожалению, такой подход не всегда даёт то, что хотелось бы. Имеются составные числа \textit{n}, обладающие свойством (***) для любого целого
\textit{a} с условием \textit{(a, n) = 1}. Такие числа называются числами Кармайкла. Рассмотрим, например, число \textit{561 = 3{$\cdot$}11{$\cdot$}17}. 
Так как 560 делится на каждое из чисел 2, 10, 16, то с помощью малой теоремы Ферма легко проверить, что 561 есть число Кармайкла. 
Можно доказать, что любое из чисел Кармайкла имеет вид \textit{n = p\textsubscript{1} {$\cdot$} {$\dots$} {$\cdot$} p\textsubscript{r}, r {$\geq$} 3}, 
где все простые \textit{p\textsubscript{i}} различны, причем \textit{n - 1} делится на каждую разность \textit{p\textsubscript{i} - 1}. 
Лишь недавно, была решена проблема о бесконечности множества таких чисел.

  В 1976 г. Миллер предложил заменить проверку (***) проверкой несколько иного условия. Если \textit{n} - простое число, 
\textit{n - 1 = 2\textsuperscript{ s} t}, где \textit{t} нечётно, то согласно малой теореме Ферма для каждого \textit{a} с 
условием \textit{(a, n) = 1} хотя бы одна из скобок в произведении
\begin{equation}
 \textit{(a\textsuperscript{ t} - 1)(a\textsuperscript{ t} + 1)(a\textsuperscript{ 2 t} + 1){$\cdot$} {$\dots$} {$\cdot$}(a\textsuperscript{ 2\textsuperscript{ s - 1}t} + 1) = a\textsuperscript{ n - 1} - 1}
\end{equation}
делится на \textit{n}. Обращение этого свойства можно использовать, чтобы отличать составные числа от простых.

Пусть \textit{n} - нечётное составное число, \textit{n - 1 = 2\textsubscript{s} t}, где \textit{t} нечётно. Назовем целое число \textit{a},
\textit{1 < a < n}, «хорошим» для \textit{n}, если нарушается одно из двух условий: 
\begin{enumerate}
 \item \textit{n} не делится на \textit{a};
 \item \textit{a\textsuperscript{ t} {$\equiv$} 1 (mod n)} или существует целое \textit{k}, 0 {$\leq$} k {$\leq$} s, такое, что
  \begin{equation}
      \textit{a\textsuperscript{ 2\textsuperscript{ k} t} {$\equiv$} -1 (mod n)}
  \end{equation}

\end{enumerate}
Из сказанного ранее следует, что для простого числа \textit{n} не существует хороших чисел \textit{a}. Если же \textit{n} составное число, 
то, как доказал Рабин, их существует не менее \textit{{$\dfrac{3}{4}$}(n - 1)}.


\subsection{Проверка большого числа на простоту}

\paragraph{} Перед описанием некоторых алгоритмов и тестов на простоту чисел введем определения и утверждения

  \textbf{Определение.} Пусть \textit{n} {$\in$} \textit{N} нечетное число и \textit{n – 1 = 2\textsuperscript{ s} t}, \textit{t} – нечетное,
число \textit{a} {$\in$} \textit{N}, такое, что \textit{0 < a < n}, называется свидетелем 
простоты числа \textit{n}, если выполнены 2 условия:

  \begin{enumerate}
    \item \textit{(a, n) = 1}
    \item Справедливо хотя бы одно {$\in$} сравнений \textit{a\textsuperscript{ t} {$\equiv$} 1 (mod n)}, 
\textit{a\textsuperscript{ t} {$\equiv$} -1 (mod n)}, \textit{a\textsuperscript{ 2 t} {$\equiv$} -1(mod n)},
{$\dots$}, \textit{a\textsuperscript{ t 2\textsuperscript{ s-1}} {$\equiv$} -1(mod n)}
  \end{enumerate}

  \textbf{Определение.} Составное число \textit{n} {$\in$} \textit{N} называется сильно псевдопростым по основанию \textit{a} 
{$\in$} \textit{Z\textsubscript{ n}}, если число \textit{a} является свидетелем простоты числа \textit{n}.

  \textbf{Утверждение.} Пусть \textit{a}, \textit{n} {$\in$} \textit{N}, \textit{0 < a < n}, если число a не является свидетелем 
простоты числа \textit{n}, то \textit{n} - составное

\subsubsection{Тест на свидетельство простоты}

\paragraph{}Пусть даны 2 числа \textit{a}, \textit{n} {$\in$} \textit{N}, \textit{n} – нечетное, \textit{n - 1 = 2\textsuperscript{ s} t}, 
\textit{t} – нечетное, необходимо проверить, является ли число \textit{а} свидетелем простоты числа \textit{n}.

  \begin{enumerate}
   \item Если \textit{(a, n) {$\ge$} 1}, то \textit{а} не является свидетелем простоты числа \textit{n}, конец алгоритма
   \item Вычисляем \textit{a\textsuperscript{ t} (mod n)}, если \textit{a\textsuperscript{ t} {$\equiv$} {$\pm$}1(mod n)}, то \textit{а}
– свидетель простоты числа \textit{n}, конец алгоритма
   \item Последовательно для \textit{i=1, {$\dots$}, s - 1} вычисляем \textit{a\textsuperscript{ t 2\textsuperscript{ i}}(mod n)} и 
проверяем, если \textit{a\textsuperscript{ t 2\textsuperscript{ i}} {$\equiv$} n - 1 (mod n)}, то \textit{а} – 
свидетель простоты числа \textit{n}, иначе \textit{а} не является свидетелем простоты числа \textit{n}
  \end{enumerate}


\subsubsection{Тест Миллер-Рабина}

\paragraph{}Пусть дано нечетное \textit{n} {$\in$} \textit{N}, \textit{n - 1 = 2\textsuperscript{ s} t}, \textit{t} – нечетное. Необходимо выяснить 
с вероятностью \textit{1 – 4\textsuperscript{ -k}}, что \textit{n} – составное.
  
  \begin{enumerate}
   \item Пусть \textit{i = 1}
   \item Случайным образом выбираем натуральное число \textit{a}, \textit{1 < a < n}
   \item Если \textit{a} не является свидетелем простоты числа \textit{n}, то \textit{n} - составное, конец алгоритма
   \item Если \textit{i = k}, то \textit{n} — простое, конец алгоритма, иначе \textit{i = i + 1} и переходим к шагу 2
  \end{enumerate}

\subsubsection{Улучшенный тест Миллер-Рабина}

\paragraph{} Пусть число n {$\ge$} 2 – нечетно и \textit{n - 1 = 2\textsuperscript{ s} d}, где \textit{d} – нечетно. Для каждого числа \textit{a}
от 2 до \textit{r + 1} , где \textit{r} – число проверок в  тесте, выполним следующие действия:

  \begin{enumerate}
   \item Вычисляем \textit{x\textsubscript{0} = a\textsuperscript{ d}(mod n)}
   \item Проверяем условие \textit{x\textsubscript{0}} {$\in$} \textit{{1, n-1}}. Если оно выполнится, тогда \textit{a} – свидетель 
простоты. Перейдем к следующему \textit{a}.
   \item Иначе проверим, содержится ли число \textit{n - 1} в последовательности 
\textit{{x\textsubscript{1}, x\textsubscript{2}, {$\dots$}, x\textsubscript{s-1}}}, где каждый последующий \textit{x} 
вычисляется по формуле \textit{x\textsubscript{i+1}=x\textsubscript{i}\textsuperscript{ 2} (mod n)}
  \end{enumerate}
  
  Если ответ положительный, то \textit{a} – свидетель простоты. Перейдем к следующему \textit{a {$\leq$} r+1}. Иначе, найден свидетель 
непростоты \textit{n}. Завершаем тест с сообщением «число n – составное». Если после \textit{r} проверок окажется \textit{r} свидетелей
простоты, то заканчиваем тест с сообщением «n – вероятно простое».

\paragraph{} \textbf{Утверждение.} Тест Миллера-Рабина определяет, что \textit{n} {$\in$} \textit{N} – простое с вероятностью
\textit{1 - 4\textsuperscript{ -k}} менее, чем за \textit{8 k N} арифметических операций, где \textit{N = {$\log_{2}{n}$} +1 }

  \textbf{Пример.} Пусть \textit{n = 1729} – число Кармайкла, разложим \textit{n - 1 = 2\textsuperscript{ 6} 3\textsuperscript{ 3}}. 
Выполним тест Миллера–Рабина для \textit{a = 2}:
  
  \begin{enumerate}
   \item x\textsubscript{0} = 2\textsuperscript{ 27} (mod 1729) = 625 {$\ne$} 1, {$\ne$} n - 1
   \item x\textsubscript{1} = x\textsubscript{0}\textsuperscript{2} (mod 1729) = 645\textsuperscript{ 2} (mod 1729) = 1065
   \item x\textsubscript{2} = x\textsubscript{1}\textsuperscript{2} (mod 1729) = 1065\textsuperscript{ 2} (mod 1729) = 1
  \end{enumerate}
  
  Последующие элементы \textit{{x\textsubscript{i}}} для \textit{i = 3, 4, 5} равны 1, и последовательность 
\textit{{x\textsubscript{1}, x\textsubscript{2}, {$\dots$}, x\textsubscript{s-1}}}, 
не содержит \textit{n - 1}. Значит, 2 является свидетелем непростоты \textit{n}, и \textit{n = 1729} – составное число

\subsection{Построение больших простых чисел}

\subsection{Практические рекомендации}

  
  
