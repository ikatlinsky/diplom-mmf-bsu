\section{Построение больших простых чисел}

\paragraph{} Существует довольно эффективный способ убедиться, что заданное число является составным, не разлагая это число на множители. 
Согласно малой теореме Ферма, если число \textit{n} простое, то для любого целого \textit{a}, не делящегося на \textit{n}, выполняется сравнение
\begin{equation}
 \textit{a\textsuperscript{ n-1} {$\equiv$} 1 (mod n)}.
\end{equation}
Если же при каком-то \textit{a} это сравнение нарушается, можно утверждать, что \textit{n} - составное. Вопрос только в том, как найти для составного \textit{n}
целое число \textit{a}, не удовлетворяющее (***). Можно, например, пытаться найти необходимое число \textit{a}, испытывая все целые числа подряд, 
начиная с \textit{2}. Или попробовать выбирать эти числа случайным образом на отрезке \textit{1 < a < n}.

  К сожалению, такой подход не всегда даёт то, что хотелось бы. Имеются составные числа \textit{n}, обладающие свойством (***) для любого целого
\textit{a} с условием \textit{(a, n) = 1}. Такие числа называются числами Кармайкла. Рассмотрим, например, число \textit{561 = 3{$\cdot$}11{$\cdot$}17}. 
Так как 560 делится на каждое из чисел 2, 10, 16, то с помощью малой теоремы Ферма легко проверить, что 561 есть число Кармайкла. 
Можно доказать, что любое из чисел Кармайкла имеет вид \textit{n = p\textsubscript{1} {$\cdot$} {$\dots$} {$\cdot$} p\textsubscript{r}, r {$\geq$} 3}, 
где все простые \textit{p\textsubscript{i}} различны, причем \textit{n - 1} делится на каждую разность \textit{p\textsubscript{i} - 1}. 
Лишь недавно, была решена проблема о бесконечности множества таких чисел.

  В 1976 г. Миллер предложил заменить проверку (***) проверкой несколько иного условия. Если \textit{n} - простое число, 
\textit{n - 1 = 2\textsuperscript{ s} t}, где \textit{t} нечётно, то согласно малой теореме Ферма для каждого \textit{a} с 
условием \textit{(a, n) = 1} хотя бы одна из скобок в произведении
\begin{equation}
 \textit{(a\textsuperscript{ t} - 1)(a\textsuperscript{ t} + 1)(a\textsuperscript{ 2 t} + 1){$\cdot$} {$\dots$} {$\cdot$}(a\textsuperscript{ 2\textsuperscript{ s - 1}t} + 1) = a\textsuperscript{ n - 1} - 1}
\end{equation}
делится на \textit{n}. Обращение этого свойства можно использовать, чтобы отличать составные числа от простых.

Пусть \textit{n} - нечётное составное число, \textit{n - 1 = 2\textsubscript{s} t}, где \textit{t} нечётно. Назовем целое число \textit{a},
\textit{1 < a < n}, «хорошим» для \textit{n}, если нарушается одно из двух условий: 
\begin{enumerate}
 \item \textit{n} не делится на \textit{a};
 \item \textit{a\textsuperscript{ t} {$\equiv$} 1 (mod n)} или существует целое \textit{k}, 0 {$\leq$} k < s, такое, что
  \begin{equation}
      \textit{a\textsuperscript{ 2\textsuperscript{ k} t} {$\equiv$} -1 (mod n)}
  \end{equation}

\end{enumerate}
Из сказанного ранее следует, что для простого числа \textit{n} не существует хороших чисел \textit{a}. Если же \textit{n} составное число, 
то, как доказал Рабин, их существует не менее \textit{{$\dfrac{3}{4}$}(n - 1)}.


\subsection{Проверка большого числа на простоту}

\paragraph{} Есть некоторое отличие в постановках задач предыдущего и настоящего пунктов. Когда мы строим простое число \textit{n}, 
мы обладаем некоторой дополнительной информацией о нем, возникающей в процессе построения. Например, такой информацией 
является знание простых делителей числа \textit{n - 1}. Эта информация иногда облегчает доказательство простоты \textit{n}.

  В настоящее время известны детерминированные алгоритмы различной сложности для доказательства простоты чисел. К примеру алгоритм
Адлемана, Померанца и Рамели. Для доказательства простоты или непростоты числа \textit{n} этот алгоритм требует \textit{({$\ln$}n)\textsuperscript{ C {$\ln$} {$\ln$} {$\ln$}n}} 
арифметических операций. Здесь \textit{C} - некоторая положительная абсолютная постоянная. Функция \textit{{$\ln$} {$\ln$} {$\ln$}n} хоть и медленно, 
но всё же возрастает с ростом \textit{n}, поэтому алгоритм не является полиномиальным. Но всё же его практические реализации позволяют достаточно 
быстро тестировать числа на простоту.

\paragraph{}

\subsection{Примеры}

\paragraph{}