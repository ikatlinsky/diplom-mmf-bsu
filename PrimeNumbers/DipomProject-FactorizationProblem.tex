\section{Проблема факторизации}

\subsection{Метод Полларда}

\paragraph{} Пусть \textit{N} – нечетное составное число, \textit{S = {0, 1, {$\dots$}, N - 1}} и \textit{f:S {$\rightarrow$} S} – случайное отображение, 
обладающее сжимающими свойствами, например \textit{f(x)=x\textsuperscript{ 2} + 1(mod N)}. Основная идея метода состоит в следующем. Выбираем случайный 
элемент \textit{x\textsubscript{0}} из \textit{S} и строим последовательность \textit{x\textsubscript{0}, x\textsubscript{1}, {$\dots$}, } 
определяемую рекуррентным соотношением \textit{x\textsubscript{i+1}=f(x\textsubscript{i})}, где \textit{i {$\ge$} 0}, до тех пор, пока не 
найдем числа \textit{i}, \textit{j}, такие что \textit{i < j} и \textit{x\textsubscript{i} = x\textsubscript{j}}. Поскольку множество \textit{S} 
конечно, такие индексы \textit{i}, \textit{j} существуют.

  Если \textit{p} - простой делитель числа \textit{N} и \textit{x\textsubscript{i} {$\equiv$} x\textsubscript{j}(mod p)}, то разность
\textit{x\textsubscript{i} - x\textsubscript{j}} делится на \textit{p} и \textit{GCD(x\textsubscript{i} - x\textsubscript{j}, N) > 1}. Нетривиальный 
наибольший делитель \textit{1 < d < N} и будет искомым делителем \textit{p} числа \textit{N}. Вероятность случая \textit{d = N} крайне мала.

  Если последовательно вычислять значения \textit{x\textsubscript{i+1}=f(x\textsubscript{i})}, запоминать их и искать равные, то в среднем число шагов 
алгоритма будет равно \textit{( {$\pi$} {$\frac{N}{2}$})\textsuperscript{ {$\frac{1}{2}$}}}. Кроме того, временная сложность алгоритма равна
\textit{O(N\textsuperscript{ {$\frac{1}{2}$}})}.

  \subsubsection{Алгоритм (метод Полларада)}
  
  Вход: Число \textit{N}, начальное значение \textit{c}, функция \textit{f}  
  Выход: нетривиальный делитель \textit{p} числа \textit{N}
  
    \begin{enumerate}
     \item Нужно положить, что \textit{a = c}, \textit{b = c}
     \item Вычислить \textit{a {$\equiv$} f(a)(mod N)}, \textit{b {$\equiv$} f(b)(mod N)}
     \item Найти \textit{d = GCD(a - b, N)}
     \item Если \textit{1 < d < N}, то положить \textit{p = d} и вернуть \textit{p}, если \textit{d = 1}, то идем на шаг 2, при \textit{d = N} 
	результат не найден, конец алгоритма
    \end{enumerate}
  
  \begin{example}
    Необходимо найти нетривиальный делитель числа \textit{n = 1373503}. На первом шаге положим \textit{c = 2}, \textit{f(x) = (x\textsuperscript{2} - 2)(mod n)}, 
    получаем, что \textit{a = b = 2} и \textit{d = 1373503 = n}, значит надо поменять отображение \textit{f(x)}, пусть оно будет в виде 
    \textit{f(x) = (x\textsuperscript{2} - 22)(mod n)}, за 5 итераций в итоге получим, что \textit{a = 777096}, \textit{b = 742847} и \textit{d = 1181}. 
    Действительно \textit{n = 1181 {$\times$} 1163}.
  \end{example}  

\subsection{Метод квадратов}

\paragraph{} Рассмотрим теорему Ферма о разложении

  \begin{theorem}
    Для любого положительного нечетного числа \textit{N} существует взаимно однозначное соответствие между множеством делителей числа \textit{N}, 
    не меньших, чем \textit{N\textsuperscript{ {$\frac{1}{2}$}}}, и множеством пар \textit{{s, t}} таких неотрицательных целых чисел, что 
    \textit{N = s\textsuperscript{ 2} - t\textsuperscript{ 2}}
  \end{theorem}

  Если \textit{N = p {$\times$} q}, где числа \textit{p} и \textit{q} близки друг к другу, то число \textit{t} мало, а значит, \textit{s} немного больше,
  чем \textit{N\textsuperscript{ {$\frac{1}{2}$}}}. В этом случае можно найти \textit{p} и \textit{q}, последовательно перебирая числа
  \textit{s=[\textit{N\textsuperscript{ {$\frac{1}{2}$}}}]+1, [\textit{N\textsuperscript{ {$\frac{1}{2}$}}}]+2, {$\dots$}} до тех пор, пока не найдется 
  такое \textit{s}, что разность \textit{s\textsuperscript{ 2} – N} является полным квадратом, то есть равна \textit{t\textsuperscript{ 2}}.
  
  \begin{example}
    Нужно разложить на множители число n = 1406303 методом Ферма. Вычисляем по шагам:
	\begin{enumerate}
	\item s = [N\textsuperscript{ {$\frac{1}{2}$}}] + 1 = 1185 + 1 = 1186, s\textsuperscript{2} - n = 1186\textsuperscript{2} - 1406303 = 293
	\item s = [N\textsuperscript{ {$\frac{1}{2}$}}] + 2 = 1185 + 2 = 1187, s\textsuperscript{2} - n = 1187\textsuperscript{2} - 1406303 = 2666
	\item s = [N\textsuperscript{ {$\frac{1}{2}$}}] + 3 = 1185 + 3 = 1188, s\textsuperscript{2} - n = 1188\textsuperscript{2} - 1406303 = 71\textsuperscript{2}
	\end{enumerate}

    Таким образом получили, что \textit{s = 1188}, \textit{t = 71}, \textit{p = s - t = 1117}, \textit{q = s + t = 1259}, как результат 
    \textit{N = 1117 {$\times$} 1259}
  \end{example}


\subsection{Обобщенный метод Ферма}

  \paragraph{} Понятно, что чем больше разность между числами \textit{p} и \textit{q}, тем более трудоемким становится метод Ферма, в этом случае 
  можно воспользоваться обобщенным методом Ферма: для небольшого целого числа \textit{k} последовательно вычислять 
  \textit{s = [(kN)\textsuperscript{ {$\frac{1}{2}$}}] + 1, [(kN)\textsuperscript{ {$\frac{1}{2}$}}] + 2, {$\dots$}} пока не получится такое число \textit{s}, 
  что разность \textit{s\textsuperscript{2} - kn} является полным квадратом, то есть равна \textit{t\textsuperscript{2}}. Отсюда
  \textit{(s + t){$\times$}(s - t) = kN}, и значит, числа \textit{s - t} и \textit{N} имеют нетривиальный общий делитель.
  
  \begin{example}
    Нужно разложить на множители число \textit{N = 5338771}. Если будем использовать стандартный метод Ферма, то разложение получим лишь на 160 шаге
    
      \begin{subequations}
	\begin{align}
	  s = [N\textsuperscript{ {$\frac{1}{2}$}}] + 160 = 2310 + 160 = 2470, \\
	  t\textsuperscript{2} = 2470\textsuperscript{2} - 5338771 = 762129, \\
	  t = 873, s + t = 3343, s - t = 1597, 
       \end{align}
      \end{subequations}	
    
    однако выбрав \textit{k = 8} решим задачу на 2ом шаге: 
      \begin{equation}
       ([(kN)\textsuperscript{ {$\frac{1}{2}$}}] + 2)\textsuperscript{2} - k n = 6537\textsuperscript{2} - 8 5338771 = 149\textsuperscript{2}
      \end{equation}	
	
    А значит мы получили, что \textit{s = 6537}, \textit{t = 149} , \textit{s - t = 6537 - 148 = 6388}, \textit{GCD(s - t, N) = 1597} и \textit{N = 1597 {$\times$} 3343}
  \end{example}


\subsection{Метод Диксона}

\paragraph{} Поскольку подобрать такое число \textit{k} не легко, на практике часто для разложения числа \textit{N} достаточно найти такие целые числа \textit{s},
\textit{t}, что \textit{s\textsuperscript{2} {$\equiv$} t\textsuperscript{2}(mod N)}, то есть такие, что \textit{(s + t)(s - t) {$\equiv$} 0 (mod N)}. 
Если \textit{s {$\equiv$} {$\pm$}t(mod N)}, то число \textit{N} делит произведение двух чисел \textit{s + t} и \textit{s - t}, но не делит ни один 
из сомножителей, значит, один делитель числа \textit{N} делит разность \textit{s - t}, а другой делитель \textit{q = {$\frac{N}{p}$}} делит сумму \textit{s + t}.
Описанный выше метод называется методом Диксона.

  \subsubsection{Алгоритм Диксона}
  Вход: составное число \textit{N}
  Выход: нетривиальный делитель \textit{p} числа \textit{N}
  
    \begin{enumerate}
     \item Построить базу разложения \textit{B = {p\textsubscript{0}, p\textsubscript{1}, {$\dots$}, p\textsubscript{h}}}, 
     где \textit{p\textsubscript{0} = -1} и \textit{p\textsubscript{1}, {$\dots$}, p\textsubscript{h}} - попарно различные простые числа.
     \item Найти \textit{h + 2} целых числа \textit{s\textsubscript{i}}, для каждого из которых абсолютно наименьший вычет
     \textit{s\textsubscript{i}\textsuperscript{2}(mod N)} является \textit{B-гладким}:
     
	\begin{equation}
	  \textit{s\textsubscript{i}\textsuperscript{2}(mod N) = {$\prod_{j=0}^{h}{p_i^\alpha}$}}
	\end{equation}
     
     \item Найти непустое множество \textit{K {$\subseteq$} (1, 2, {$\dots$}, h+1)}, такое что \textit{{$\oplus$}e\textsubscript{k} = 0}, где 
     
	\begin{equation}
	  \textit{{$e_k = (e_1^k, e_2^k, \dots, e_h^k), e_j^k \equiv \alpha_j(mod 2), 0 \le j \le h$}}
	\end{equation}
     
     \item Положить 
     
	\begin{equation}
	  \textit{{$ s = \prod_{k \in K}{s_k(mod N)}, t = \prod_{i=1}^{h}{p_i\textsuperscript{{$\frac{1}{2}$} {$\sum_{k \in K}{\alpha_i}$}}(mod N)}$},}
	\end{equation}
     
     тогда \textit{s\textsuperscript{2} {$\equiv$} t\textsuperscript{2}(mod N)}.
     \item Если \textit{s {$\equiv$} {$\pm$}t(mod N)}, то \textit{p = GCD(s - t, N)} и возвращаем \textit{p}, иначе возвращаемся на шаг 3 и меняем множество \textit{K}
    \end{enumerate}
  

\subsection{Метод квадратичного решета}

\paragraph{} Модификация выбора чисел \textit{s\textsubscript{i}} в алгоритме Диксона была разработана К. Померанцем в 1981 году. Долгое время 
превосходил другие методы факторизации целых чисел общего вида, не имеющих простых делителей, порядок которых значительно меньше \textit{{$\sqrt{N}$}} 
(для чисел \textit{N}, имеющих простые делители, много меньшие более быстрым является метод факторизации на эллиптических кривых). 
Метод квадратичного решета представляет собой разновидность метода факторных баз (обобщение метода факторизации Ферма). 
Этот метод считается вторым по быстроте (после общего метода решета числового поля). И до сих пор является самым быстрым для 
целых чисел до 100 десятичных цифр и устроен значительно проще чем общий метод решета числового поля. Это универсальный алгоритм факторизации, 
так как время его выполнения исключительно зависит от размера факторизуемого числа, а не от его особой структуры и свойств.

  В этой модификации мы рассматриваем функцию
    
    \begin{equation}
      \textit{f(x) = (x + [{$\sqrt{N}$}])\textsuperscript{2} - N = x\textsuperscript{2} + 2[{$\sqrt{N}$}]x + [{$\sqrt{N}$}]\textsuperscript{2} - N}
    \end{equation}
    
  Поскольку \textit{f(x)} эквивалентно \textit{x\textsuperscript{2} + 2[{$\sqrt{N}$}]x}, при малых значениях \textit{x}, \textit{-c {$\le$} x {$\le$} c}, 
  где \textit{c} – некоторая константа, значение \textit{f(x)} тоже мало, и с большой вероятностью легко раскладывается на множители, то есть в 
  качестве \textit{s\textsubscript{i}} можно брать значения \textit{f(x)}, где \textit{-c {$\le$} x {$\le$} c}.
  
  Если \textit{f(x)} делится на простое число \textit{p\textsubscript{i} {$\ne$} 2} для некоторого целого \textit{x}, то сравнение
  
    \begin{equation}
      \textit{f(x) = (x + [{$\sqrt{N}$}])\textsuperscript{2} {$\equiv$} N(mod p\textsubscript{i})}
    \end{equation}
    
  Выполняется только в том случае если \textit{N} – квадратичный вычет по модулю \textit{p\textsubscript{i}}, то есть в базу разложения следует включать 
  только те нечетные простые числа \textit{p\textsubscript{i}}, для которых \textit{{$\binom{N}{p\textsubscript{i}}$} = 1}.
  
  Чтобы числа \textit{f(x)} были с большей вероятностью \textit{B-гладкими}, осуществляется просеивание: решая сравнение (2.2.7) относительно \textit{x\textsubscript{2}}
  каждого простого числа \textit{p\textsubscript{i} {$\ne$} 2} из базы разложения, в результате получим 2 решения \textit{x\textsubscript{1}} и \textit{x\textsubscript{2}}. 
  Тогда все целые числа вида \textit{x\textsubscript{j}(p\textsubscript{i}) + k p\textsubscript{i}} будут удовлетворять сравнению. Таким образом, в 
  интервале \textit{[-c, c]} можно оставить только те числа \textit{x}, которые удовлетворяют сравнению \textit{x {$\equiv$} x\textsubscript{1}(p\textsubscript{i})(mod p\textsubscript{i})} 
  или \textit{x {$\equiv$} x\textsubscript{2}(p\textsubscript{i})(mod p\textsubscript{i})} для достаточно большого числа элементов 
  \textit{p\textsubscript{i}} из \textit{B}.
  
  Оптимальное значение \textit{h} в этом алгоритме \textit{O(exp({$\frac{1}{2}$}{$\sqrt{\ln{n}\ln{\ln{n}}}$}))}. Временная сложность 
  алгоритма равна \textit{O(exp({$\sqrt{\ln{n}\ln{\ln{n}}}$}))}.

\subsection{Метод решета числового поля}