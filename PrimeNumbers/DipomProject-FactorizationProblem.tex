\section{Проблема факторизации}

\subsection{Метод Полларда}

\paragraph{} Пусть \textit{N} – нечетное составное число, \textit{S = {0, 1, {$\dots$}, N - 1}} и \textit{f:S {$\rightarrow$} S} – случайное отображение, 
обладающее сжимающими свойствами, например \textit{f(x)=x\textsuperscript{ 2} + 1(mod N)}. Основная идея метода состоит в следующем. Выбираем случайный 
элемент \textit{x\textsubscript{0}} из \textit{S} и строим последовательность \textit{x\textsubscript{0}, x\textsubscript{1}, {$\dots$}, } 
определяемую рекуррентным соотношением \textit{x\textsubscript{i+1}=f(x\textsubscript{i})}, где \textit{i {$\ge$} 0}, до тех пор, пока не 
найдем числа \textit{i}, \textit{j}, такие что \textit{i < j} и \textit{x\textsubscript{i} = x\textsubscript{j}}. Поскольку множество \textit{S} 
конечно, такие индексы \textit{i}, \textit{j} существуют.

  Если \textit{p} - простой делитель числа \textit{N} и \textit{x\textsubscript{i} {$\equiv$} x\textsubscript{j}(mod p)}, то разность
\textit{x\textsubscript{i} - x\textsubscript{j}} делится на \textit{p} и \textit{GCD(x\textsubscript{i} - x\textsubscript{j}, N) > 1}. Нетривиальный 
наибольший делитель \textit{1 < d < N} и будет искомым делителем \textit{p} числа \textit{N}. Вероятность случая \textit{d = N} крайне мала.

  Если последовательно вычислять значения \textit{x\textsubscript{i+1}=f(x\textsubscript{i})}, запоминать их и искать равные, то в среднем число шагов 
алгоритма будет равно \textit{( {$\pi$} {$\frac{N}{2}$})\textsuperscript{ {$\frac{1}{2}$}}}. Кроме того, временная сложность алгоритма равна
\textit{O(N\textsuperscript{ {$\frac{1}{2}$}})}.

  \subsubsection{Алгоритм (метод Полларада)}
  
  Вход: Число \textit{N}, начальное значение \textit{c}, функция \textit{f}  
  Выход: нетривиальный делитель \textit{p} числа \textit{N}
  
    \begin{enumerate}
     \item Нужно положить, что \textit{a = c}, \textit{b = c}
     \item Вычислить \textit{a {$\equiv$} f(a)(mod N)}, \textit{b {$\equiv$} f(b)(mod N)}
     \item Найти \textit{d = GCD(a - b, N)}
     \item Если \textit{1 < d < N}, то положить \textit{p = d} и вернуть \textit{p}, если \textit{d = 1}, то идем на шаг 2, при \textit{d = N} 
	результат не найден, конец алгоритма
    \end{enumerate}
  
  \begin{example}
    Необходимо найти нетривиальный делитель числа \textit{n = 1373503}. На первом шаге положим \textit{c = 2}, \textit{f(x) = (x\textsuperscript{2} - 2)(mod n)}, 
    получаем, что \textit{a = b = 2} и \textit{d = 1373503 = n}, значит надо поменять отображение \textit{f(x)}, пусть оно будет в виде 
    \textit{f(x) = (x\textsuperscript{2} - 22)(mod n)}, за 5 итераций в итоге получим, что \textit{a = 777096}, \textit{b = 742847} и \textit{d = 1181}. 
    Действительно \textit{n = 1181 {$\times$} 1163}.
  \end{example}  

\subsection{Метод квадратов}

\paragraph{} Рассмотрим теорему Ферма о разложении

  \begin{theorem}
    Для любого положительного нечетного числа \textit{N} существует взаимно однозначное соответствие между множеством делителей числа \textit{N}, 
    не меньших, чем \textit{N\textsuperscript{ {$\frac{1}{2}$}}}, и множеством пар \textit{{s, t}} таких неотрицательных целых чисел, что 
    \textit{N = s\textsuperscript{ 2} - t\textsuperscript{ 2}}
  \end{theorem}

  Если \textit{N = p {$\times$} q}, где числа \textit{p} и \textit{q} близки друг к другу, то число \textit{t} мало, а значит, \textit{s} немного больше,
  чем \textit{N\textsuperscript{ {$\frac{1}{2}$}}}. В этом случае можно найти \textit{p} и \textit{q}, последовательно перебирая числа
  \textit{s=[\textit{N\textsuperscript{ {$\frac{1}{2}$}}}]+1, [\textit{N\textsuperscript{ {$\frac{1}{2}$}}}]+2, {$\dots$}} до тех пор, пока не найдется 
  такое \textit{s}, что разность \textit{s\textsuperscript{ 2} – N} является полным квадратом, то есть равна \textit{t\textsuperscript{ 2}}.
  
  \begin{example}
    Нужно разложить на множители число n = 1406303 методом Ферма. Вычисляем по шагам:
	\begin{enumerate}
	\item s = [N\textsuperscript{ {$\frac{1}{2}$}}] + 1 = 1185 + 1 = 1186, s\textsuperscript{2} - n = 1186\textsuperscript{2} - 1406303 = 293
	\item s = [N\textsuperscript{ {$\frac{1}{2}$}}] + 2 = 1185 + 2 = 1187, s\textsuperscript{2} - n = 1187\textsuperscript{2} - 1406303 = 2666
	\item s = [N\textsuperscript{ {$\frac{1}{2}$}}] + 3 = 1185 + 3 = 1188, s\textsuperscript{2} - n = 1188\textsuperscript{2} - 1406303 = 71\textsuperscript{2}
	\end{enumerate}

    Таким образом получили, что \textit{s = 1188}, \textit{t = 71}, \textit{p = s - t = 1117}, \textit{q = s + t = 1259}, как результат 
    \textit{N = 1117 {$\times$} 1259}
  \end{example}


\subsection{Обобщенный метод Ферма}

  \paragraph{} Понятно, что чем больше разность между числами \textit{p} и \textit{q}, тем более трудоемким становится метод Ферма, в этом случае 
  можно воспользоваться обобщенным методом Ферма: для небольшого целого числа \textit{k} последовательно вычислять 
  \textit{s = [(kN)\textsuperscript{ {$\frac{1}{2}$}}] + 1, [(kN)\textsuperscript{ {$\frac{1}{2}$}}] + 2, {$\dots$}} пока не получится такое число \textit{s}, 
  что разность \textit{s\textsuperscript{2} - kn} является полным квадратом, то есть равна \textit{t\textsuperscript{2}}. Отсюда
  \textit{(s + t){$\times$}(s - t) = kN}, и значит, числа \textit{s - t} и \textit{N} имеют нетривиальный общий делитель.
  
  \begin{example}
    Нужно разложить на множители число \textit{N = 5338771}. Если будем использовать стандартный метод Ферма, то разложение получим лишь на 160 шаге
    
      \begin{subequations}
	\begin{align}
	  s = [N\textsuperscript{ {$\frac{1}{2}$}}] + 160 = 2310 + 160 = 2470, \\
	  t\textsuperscript{2} = 2470\textsuperscript{2} - 5338771 = 762129, \\
	  t = 873, s + t = 3343, s - t = 1597, 
       \end{align}
      \end{subequations}	
    
    однако выбрав \textit{k = 8} решим задачу на 2ом шаге: 
      \begin{equation}
       ([(kN)\textsuperscript{ {$\frac{1}{2}$}}] + 2)\textsuperscript{2} - k n = 6537\textsuperscript{2} - 8 5338771 = 149\textsuperscript{2}
      \end{equation}	
	
    А значит мы получили, что \textit{s = 6537}, \textit{t = 149} , \textit{s - t = 6537 - 148 = 6388}, \textit{GCD(s - t, N) = 1597} и \textit{N = 1597 {$\times$} 3343}
  \end{example}


\subsection{Метод Диксона}

\paragraph{} Поскольку подобрать такое число \textit{k} не легко, на практике часто для разложения числа \textit{N} достаточно найти такие целые числа \textit{s},
\textit{t}, что \textit{s\textsuperscript{2} {$\equiv$} t\textsuperscript{2}(mod N)}, то есть такие, что \textit{(s + t)(s - t) {$\equiv$} 0 (mod N)}. 
Если \textit{s {$\equiv$} {$\pm$}t(mod N)}, то число \textit{N} делит произведение двух чисел \textit{s + t} и \textit{s - t}, но не делит ни один 
из сомножителей, значит, один делитель числа \textit{N} делит разность \textit{s - t}, а другой делитель \textit{q = {$\frac{N}{p}$}} делит сумму \textit{s + t}.
Описанный выше метод называется методом Диксона.

  \subsubsection{Алгоритм Диксона}
  Вход: составное число \textit{N}
  Выход: нетривиальный делитель \textit{p} числа \textit{N}
  
    \begin{enumerate}
     \item Построить базу разложения \textit{B = {p\textsubscript{0}, p\textsubscript{1}, {$\dots$}, p\textsubscript{h}}}, 
     где \textit{p\textsubscript{0} = -1} и \textit{p\textsubscript{1}, {$\dots$}, p\textsubscript{h}} - попарно различные простые числа.
     \item Найти \textit{h + 2} целых числа \textit{s\textsubscript{i}}, для каждого из которых абсолютно наименьший вычет
     \textit{s\textsubscript{i}\textsuperscript{2}(mod N)} является \textit{B-гладким}:
     
	\begin{equation}
	  \textit{s\textsubscript{i}\textsuperscript{2}(mod N) = {$\prod_{j=0}^{h}{p_i^\alpha}$}}
	\end{equation}
     
     \item Найти непустое множество \textit{K {$\subseteq$} (1, 2, {$\dots$}, h+1)}, такое что \textit{{$\oplus$}e\textsubscript{k} = 0}, где 
     
	\begin{equation}
	  \textit{{$e_k = (e_1^k, e_2^k, \dots, e_h^k), e_j^k \equiv \alpha_j(mod 2), 0 \le j \le h$}}
	\end{equation}
     
     \item Положить 
     
	\begin{equation}
	  \textit{{$ s = \prod_{k \in K}{s_k(mod N)}, t = \prod_{i=1}^{h}{p_i\textsuperscript{{$\frac{1}{2}$} {$\sum_{k \in K}{\alpha_i}$}}(mod N)}$},}
	\end{equation}
     
     тогда \textit{s\textsuperscript{2} {$\equiv$} t\textsuperscript{2}(mod N)}.
     \item Если \textit{s {$\equiv$} {$\pm$}t(mod N)}, то \textit{p = GCD(s - t, N)} и возвращаем \textit{p}, иначе возвращаемся на шаг 3 и меняем множество \textit{K}
    \end{enumerate}
  

\subsection{Метод квадратичного решета}

\paragraph{} Модификация выбора чисел \textit{s\textsubscript{i}} в алгоритме Диксона была разработана К. Померанцем в 1981 году. Долгое время 
превосходил другие методы факторизации целых чисел общего вида, не имеющих простых делителей, порядок которых значительно меньше \textit{{$\sqrt{N}$}} 
(для чисел \textit{N}, имеющих простые делители, много меньшие более быстрым является метод факторизации на эллиптических кривых). 
Метод квадратичного решета представляет собой разновидность метода факторных баз (обобщение метода факторизации Ферма). 
Этот метод считается вторым по быстроте (после общего метода решета числового поля). И до сих пор является самым быстрым для 
целых чисел до 100 десятичных цифр и устроен значительно проще чем общий метод решета числового поля. Это универсальный алгоритм факторизации, 
так как время его выполнения исключительно зависит от размера факторизуемого числа, а не от его особой структуры и свойств.

  В этой модификации мы рассматриваем функцию
    
    \begin{equation}
      \textit{f(x) = (x + [{$\sqrt{N}$}])\textsuperscript{2} - N = x\textsuperscript{2} + 2[{$\sqrt{N}$}]x + [{$\sqrt{N}$}]\textsuperscript{2} - N}
    \end{equation}
    
  Поскольку \textit{f(x)} эквивалентно \textit{x\textsuperscript{2} + 2[{$\sqrt{N}$}]x}, при малых значениях \textit{x}, \textit{-c {$\le$} x {$\le$} c}, 
  где \textit{c} – некоторая константа, значение \textit{f(x)} тоже мало, и с большой вероятностью легко раскладывается на множители, то есть в 
  качестве \textit{s\textsubscript{i}} можно брать значения \textit{f(x)}, где \textit{-c {$\le$} x {$\le$} c}.
  
  Если \textit{f(x)} делится на простое число \textit{p\textsubscript{i} {$\ne$} 2} для некоторого целого \textit{x}, то сравнение
  
    \begin{equation}
      \textit{f(x) = (x + [{$\sqrt{N}$}])\textsuperscript{2} {$\equiv$} N(mod p\textsubscript{i})}
    \end{equation}
    
  Выполняется только в том случае если \textit{N} – квадратичный вычет по модулю \textit{p\textsubscript{i}}, то есть в базу разложения следует включать 
  только те нечетные простые числа \textit{p\textsubscript{i}}, для которых \textit{{$\binom{N}{p\textsubscript{i}}$} = 1}.
  
  Чтобы числа \textit{f(x)} были с большей вероятностью \textit{B-гладкими}, осуществляется просеивание: решая сравнение (2.2.7) относительно \textit{x\textsubscript{2}}
  каждого простого числа \textit{p\textsubscript{i} {$\ne$} 2} из базы разложения, в результате получим 2 решения \textit{x\textsubscript{1}} и \textit{x\textsubscript{2}}. 
  Тогда все целые числа вида \textit{x\textsubscript{j}(p\textsubscript{i}) + k p\textsubscript{i}} будут удовлетворять сравнению. Таким образом, в 
  интервале \textit{[-c, c]} можно оставить только те числа \textit{x}, которые удовлетворяют сравнению \textit{x {$\equiv$} x\textsubscript{1}(p\textsubscript{i})(mod p\textsubscript{i})} 
  или \textit{x {$\equiv$} x\textsubscript{2}(p\textsubscript{i})(mod p\textsubscript{i})} для достаточно большого числа элементов 
  \textit{p\textsubscript{i}} из \textit{B}.
  
  Оптимальное значение \textit{h} в этом алгоритме \textit{O(exp({$\frac{1}{2}$}{$\sqrt{\ln{n}\ln{\ln{n}}}$}))}. Временная сложность 
  алгоритма равна \textit{O(exp({$\sqrt{\ln{n}\ln{\ln{n}}}$}))}.

\subsection{Метод решета числового поля}

\paragraph{} \textit{Метод решета числового поля} - очень сложный алгоритм факторизации чисел, в то же время он позволяет успешно факторизовать числа
  достаточно быстро, однако из-за большого числа сложных шагов, этого алгоритм становится очень медленным и не эффективным при факторизации
  не больших чисел, тем не менее при факторизации больших чисел время, необходимое для выполнения предварительных шагов ничтожно и алгоритм работает
  значительно быстрее аналогов. Этот факт был доказан в 2002 году, когда менее чем за 4 месяца было разложено число, состоящее из 155
  знаков - 512 бит, если бы это число было публичным ключом в \textit{RSA}, то система была бы взломана. Разработка метода решета числового поля 
  поставила под вопрос эффективность крипотсистемы \textit{RSA}.
  
  Метод решета числового поля (как специальный, так и общий) можно представить как усовершенствование более простого метода - метода 
  рационального решета, либо метода квадратичного решета. Подобные им алгоритмы требуют нахождение гладких чисел порядка {$\sqrt{N}$}. 
  Размер этих чисел экспоненциально растёт с ростом {$N$}. Метод решета числового поля, в свою очередь, требует нахождения гладких чисел 
  субэкспоненциального относительно {$N$} размера. Благодаря тому, что эти числа меньше, вероятность того, что число такого размера окажется гладким выше, что и 
  является причиной эффективности метода решета числового поля. Для достижения ускорения вычислений в рамках метода проводятся в числовых 
  полях, что усложняет алгоритм, по сравнению с более простым рациональным решетом.
  
  Основные принципы:
    \begin{itemize}
     \item Метод факторизации Ферма для факторизации натуральных нечетных чисел {$n$}, состоящий в поиске таких целых чисел {$x$} и {$y$},
      что {$x^2-y^2-n$}, что ведет к разложению {$n=(x-y)(x+y)$}.
     \item Нахождение подмножества множества целых чисел, произведение которых - квадрат
     \item Составление факторной базы: набора {$\{-1, p_1, \dots, p_n \}$}, где {$p_i$} - простые числа такие, что {$p_i \le B$} 
      для некоторого {$B$}.
     \item Просеивание выполняется подобно решету Эратосфена (откуда метод и получил своё название). Решетом служат простые числа 
      факторной базы и их степени. При просеивании число не «вычёркиваются», а делится на число из решета. Если в результате число 
      оказалось единицей, то оно \textit{B-гладкое}.
     \item Основная идея состоит в том, чтобы вместо перебора чисел и проверки, делятся ли их квадраты по модулю n на простые числа из 
      факторной базы, перебираются простые числа из базы и сразу для всех чисел вида {$x^2-n$} проверяется, делятся ли они на это простое 
      число или его степень.
    \end{itemize}      
    
  \subsubsection{Алгоритм решета числового поля}
  Пусть {$n$} - нечетное составное число, которое требуется факторизовать.
  
    \begin{enumerate}
      \item Выберем степень неприводимого многочлена {$d \ge 3$} (при {$d = 2$} не будет выигрыша в сравнении с методом квадратичного решета).
      \item Выберем целое {$m$} такое, что {$\lfloor n^\frac{1}{d+1} \rfloor < m < \lfloor n^\frac{1}{d} \rfloor$}, и разложим n по 
	основанию {$m$}:
	
	  \begin{equation}
	   \textit{{$ n = m^d + a_{d-1}m^{d-1} + \dots + a_0 $}}
	  \end{equation}

      \item Свяжем с разложением (2.2.8) неприводимый в кольце {$Z[x]$} полиномов с целыми коэффициентами многочлен
      
	\begin{equation}
	 \textit{{$ f_1(x)=x^d + a_{d-1}x^{d-1} + \dots + a_0 $}}
	\end{equation}

      \item Определим полином просеивания {$F_1(a,b)$} как однородный многочлен от двух переменных {$a$} и {$b$}:
      
	\begin{equation}
	 \textit{{$ F_1(a,b) = b^d f_1(a/b) = a^d + a_{d-1}a^{d-1}b + \dots + a_0 b^d $}}
	\end{equation}

      \item Определим второй полином и соответствующий однородный многочлен {$F_2(a,b)=a-bm, f_2(x)=x-m$} .
      \item Выберем два положительных числа {$L_1$} и {$L_2$}, определяющих область просеивания:
      
	\begin{equation}
	 \textit{{$ SR = \{ 1 \le b \le L_1, -L_2 \le a \le L_2 \} $}}
	\end{equation}

      \item Пусть {$\theta$} - корень {$f_1(x)$}. Рассмотрим кольцо полиномов {$Z[\theta]$}. Определим множество, называемое алгебраической 
	факторной базой {$FB_1$}, состоящее из многочленов первого порядка вида {$a-b\theta$} с нормой (2.2.10), являющейся простым числом. 
	Эти многочлены - простые неразложимые в кольце алгебраических целых поля {$K=Q[\theta]$}. Ограничим абсолютные значения норм 
	полиномов из {$FB_1$} константой {$L_1$}.
	
      \item Определим рациональную факторную базу {$FB_2$}, состоящую из всех простых чисел, ограниченных сверху константой {$L_2$}.
      \item Определим множество {$FB_3$}, называемое факторной базой квадратичных характеров. Это множество полиномов первого порядка
	{$c-d\theta$}, норма которых - простое число. Должно выполняться условие {$FB_1 \bigcap FB_3 = \O{}$}.
      \item Выполним просеивание многочленов {$\{ a - b\theta \;|\; (a,b) \in SR \}$} по факторной базе {$FB_1$} и целых чисел
	{$\{ a-bm \;|\; (a,b) \in SR \}$} по факторной базе {$FB_2$}. В результате получим множество {$M$}, состоящее из гладких пар {$(a,b)$}, 
	то есть таких пар {$(a,b)$}, что {$НОД(a,b) = 1$}, полином и число {$a-b\theta$} и {$a-bm$} раскладываются полностью по {$FB_1$} и
	{$FB_2$} соответственно.
      \item Найдём такое подмножество {$S \subseteq M$}, что
      
	\begin{equation}
	 \textit{{$ \prod_{(a,b) \in S}{(a-b\theta)}=H^2, H \in Z; \prod_{(a,b) \in S}{(a-bm)}=B^2, B \in Z $}}
	\end{equation}

      \item Определим многочлен
      
	\begin{equation}
	  \textit{{$ g(\theta)= f'\textsuperscript{2}_1(\theta) \prod_{(a,b) \in S}{(a-b\theta)} $}}
	\end{equation}
	где {$f'_1(x)$} - производная {$f_1(x)$}

      \item Многочлен {$g(\theta)$} является полным квадратом в кольце полиномов {$Z[\theta]$}. Пусть тогда {$\alpha(\theta)$} есть корень
	из {$g(\theta)$} и {$B$} - корень из {$B^2$}.
      \item Строим отображение {$\phi:\theta \rightarrow m$}, заменяя полином {$\alpha(\theta)$} числом {$\alpha(m)$}. 
	Это отображение является кольцевым гомоморфизмом кольца алгебраических целых чисел {$Z_k$} в кольцо {$Z$}, откуда получаем 
	соотношение:
	
	  \begin{subequations}
	    \begin{center}
	      {$A^2 = g(m)^2 \equiv \phi(g(\alpha))^2 \equiv$} \\
	      {$\equiv \phi(f'^2_1(\theta)\prod_{(a,b \in S)}{(a-b\theta)}) \equiv$} \\
	      {$\equiv f'^2_1(m)\prod_{(a,b \in S)}{(a-bm)} \equiv$} \\
	      {$\equiv f'^2_1(m) C^2 (mod \: n)$}
	    \end{center}
	  \end{subequations}

      \item Пусть {$B=f'(m)C$}. Найдём пару чисел {$(A, B)$} таких, что {$A \equiv B(mod n)$}. Тогда найдём делитель числа {$n$},
	вычисляя {$НОД(n, A \pm B)$}, как это делается, например, в методе квадратичного решета.
    \end{enumerate}

  
\section{Выводы}

\paragraph{} В настоящее время исследования в области построения быстрых алгоритмов факторизации интенсивно ведутся во всем мире. Ежегодно 
  проводятся десятки конференций по этой тематике, достигаются новые рекорды факторизации длинных чисел, исследуются известные проблемы 
  алгоритмической теории чисел и ставятся новые проблемы. Недавно (в конце 2009 г.) коллективом европейских ученых, возглавляемым Торштеном
  Кляйнъюнгом, был установлен новый рекорд по разложению 768-битового натурального числа с помощью метода решета числового поля. Предыдущий
  рекорд в 512-бит был установлен в 2000 г., т.е. переход от 512-битовых к 768-битовым числам потребовал почти 10 лет. Поэтому следующий 
  рекорд в 1024 бита при сохранении прежних темпов роста исследований планируется выполнить не ранее, чем в 2020 г.
  
  В данный ммоент известны следующие алгоритмы факторизации (некоторые были рассмотрены в рамках данной работы),
  вторая группа - субэкспоненциальные алгоритмы, это алгоритмы, которые работают более, чем за полиномиальное время («сверх-полиномиальное»),
  но менее, чем за экспоненциальное время («субэкспоненциальное»). Для обозначения их сложности принята \textit{L-нотация}:  
    \begin{equation}
     \textit{{$ L_N(\alpha,c)=O(exp((c+o(1))(logN)^\alpha(loglogN)\textsuperscript{{$1-\alpha$}})) $}},
    \end{equation}    
  где {$N$} - число, подлежащее факторизации, {$0 < \alpha < 1$}  и {$c$} - некоторые константы.
  
    \begin{enumerate}
     \item Экспоненциальные алгоритмы
     
	\begin{enumerate}
	 \item перебор возможных делителей - {$O(N\textsuperscript{{$\frac{1}{2}$}})$}
	 \item метод факторизации Ферма - {$O(N\textsuperscript{{$\frac{1}{4}$}})$}
	 \item {$\rho$}-алгоритм Полларда - {$O(N\textsuperscript{{$\frac{1}{4}$}})$}
	 \item метод квадратичных форм Шенкса - {$O(N\textsuperscript{{$\frac{1}{4}$}})$}
	 \item метод Лемана - {$O(N\textsuperscript{{$\frac{1}{3}$}})$}
	\end{enumerate}
     
     \item Субэкспоненциальные алгоритмы
     
	\begin{enumerate}
	 \item алгоритм Диксона - {$L_N(\frac{1}{2}, 2\sqrt{2})$}
	 \item метод непрерывных дробей - {$L_N(\frac{1}{2}, \sqrt{2})$}
	 \item метод квадратичного решета - {$L_N(\frac{1}{2}, 1)$}	 
	 \item метод эллиптических кривых - {$L_p(\frac{1}{2}, \sqrt{2})$}, где {$p$} - наименьшее простое, которое делит {$N$}
	\end{enumerate}

    \end{enumerate}
    
    В настоящее время самыми эффективными алгоритмами факторизации являются вариации решета числового поля:
    
      \begin{enumerate}
       \item Специальный метод решета числового поля со сложностью {$L_N(\frac{1}{3}, (\frac{32}{9})^\frac{1}{3})$} 
	(метод применим для факторизации чисел только специального вида);
       \item Общий метод решета числового поля со сложностью {$L_N(\frac{1}{3}, (\frac{64}{9})^\frac{1}{3})$} 
	(метод применим ко всем числам).
      \end{enumerate}      
      
  Алгоритм Ленстры на практике часто используется для выявления (отбрасывания) небольших простых делителей числа. Однако если мы 
  работаем с числом, содержащим в себе крупные простые множители, то нам нужно увеличивать количество кривых, так как при увеличении 
  количества кривых шансы найти простой делитель возрастают, тем не менее, зависимость ожидаемого количества эллиптических кривых от 
  количества цифр в искомом делителе экспоненциальна. Метод Полларда очень быстро находит простые факторы малой и средней величины, однако,
  столкнувшись с крупным простым сомножителем, становится малоэффективным.
  
  Среди алгоритмов факторизации с экспоненциальной сложностью метод квадратичных форм Шенкса считается одним из самых эффективных. 
  Этот алгоритм работает с целыми числами, не превосходящими {$2\sqrt{N}$}. Мы знаем, что для 32-разрядных компьютеров алгоритмы, основанные на данном 
  методе, являются безусловными лидерами из алгоритмов факторизации для чисел между {$10^{10}$} до {$10^{18}$} и, вероятно, таковым и останутся. 
  Данный алгоритм может разделить практически любое составное 18-значное число менее чем за миллисекунду. Алгоритм является чрезвычайно 
  простым, красивым и эффективным. Кроме того, методы, базирующиеся на данном алгоритме, используются как вспомогательные при разложении 
  делителей больших чисел.
  
  Самым простым и самым трудоёмким экспоненциальным методом, реализуемым вручную, является подбор делителей. Он заключается в переборе 
  потенциальных делителей числа в пределах от самого числа до его квадрата. Для разложения небольших чисел он вполне подходит, так как 
  очень просто реализуется, но для разложения массивных чисел он практически непригоден.
  
  Применение на практике различных методов разложения чисел показало, что время выполнения алгоритма напрямую зависит от его типа и 
  вычислительной сложности. В данной работе мы рассмотрели некоторые алгоритмы факторизации натуральных чисел, а также провели сравнительную
  характеристику оных по группам структур. На практических примерах мы убедились в эффективности и целесообразности применения одних 
  методов перед другими, опять же с учётом структуры данного числа.
  
  Таким образом, можно сделать следующие выводы: для факторизации натуральных чисел существует достаточно много способов, но эта задача
  далеко не тривиальная, и довольно затратна в плане времени, об этом говорит оценка сложности алгоритмов факторизации. Некоторые 
  алгоритмы могут решать поставленную задачу веками. Для того, чтобы уменьшить время ожидания, нужно подобрать соответствующий структуре
  числа метод факторизации.
  
  Но решение данной проблемы не стоит на месте, поскольку многие крупные открытия и новые достижения в этой области появляются снова и снова.
  Эти достижения обусловлены не только развитием вычислительной мощности компьютеров и сетей, но и развитием тех областей теоретической 
  математики, которые до недавнего времени служили областью интересов лишь профессиональных математиков-специалистов в абстрактной алгебре и
  теории чисел.
 
