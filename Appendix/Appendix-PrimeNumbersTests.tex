% !TEX root = ../DiplomProject.tex

\chapter{Тесты чисел на простоту}

\section{Тест на свидетельство простоты}

  Дано: {$n$} - исследуемое число, {$k$} - свидель простоты \\
  Возвращает: {$True$} - если {$k$} простое, иначе - {$False$}

\begin{lstlisting}[language=Mathematica,caption={Тест Миллера-Рабина}]
  MillerRabinWitnessQ[n_, k_] := Module[{d = n - 1, s = 0, test = True, x, a, r},
    While[Mod[d, 2] == 0, d /= 2; s++];
    
    x = PowerMod[k, d, n];
    If[x != 1,
    For[r = 0, r < s, r++,
      If[x == n - 1, Continue[]];
      x = Mod[x x, n];
      ];
    If[x != n - 1, test = False];
    ];

    Return[test]
  ]
\end{lstlisting}

\section{Вероятностный тест Миллера-Рабина}

  Дано: {$n$} - исследуемое число, {$m$} - число тестов \\
  Возвращает: {$True$} - если простое, иначе - {$False$}

\begin{lstlisting}[language=Mathematica,caption={Вероятностный тест Миллера-Рабина}]
  ProbabilityMillerRabinWitnessQ[n_, m_] := Module[{a, i, tmp = {}},
    a = Table[RandomInteger[{2, n - 1}], {i, 1, m}];
    For[i = 1, i <= Length@a, i++,
      tmp = Append[tmp, If[MillerRabinWitnessQ[n, a[[i]]] == True, 1, 0]];
    ];
    If[tmp == Table[1, {Length@a}], True, False]
  ] 
\end{lstlisting}
