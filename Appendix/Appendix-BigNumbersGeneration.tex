% !TEX root = ../DiplomProject.tex

\chapter{Генерация больших чисел}

  \section{Генерация больших простых чисел}

    Дано: {$digits$} - необходимое число знаков в простом числе \\
    Возвращает: простое число с  {$digits$} знаками

    \begin{lstlisting}[language=Mathematica,caption={
      Генерация больших простых чисел
    }]
    GoodRSAPrimeQ[n_Integer, l_Integer] := PrimeQ[n] && Module[{d = 1, m = (n - 1)/2},
	     While[d <= l && ! PrimeQ[m/d], d++];
	     Return[ d <= l]
	  ]

    RSAPrime[digits_Integer] := Module[{l, cand},
        cand = RandomInteger[{10^(digits - 1), 10^digits - 1}];
        l = 10^Floor[1/30 Log[10.`, cand]];
        If[EvenQ[cand], cand++];
        While[! GoodRSAPrimeQ[cand, l], cand += 2];
        
        Return@cand
      ]

    \end{lstlisting}

    \subsection{Примеры}

      \begin{lstlisting}[language=Mathematica,caption={Пример 1}]
      p = RSAPrime[60]
      PrimeQ@p

      196315822522241188643014752489528175135046890697463808010039
      True
      \end{lstlisting}

      \begin{lstlisting}[language=Mathematica,caption={Пример 2}]
      q = RSAPrime[60]
      PrimeQ@q

      945114003366975026340821154307727591775513848718061709474219
      True
      \end{lstlisting}

  \section{Генерация с использованием теста Миллера-Рабина}

    Дано: {$num$} - интервал для числа {$[1;num]$}, {$m$} - число тестов \\
    Возвращает: простое число {$\in$} {$[1;num]$}

    \begin{lstlisting}[language=Mathematica,caption={
      Генерация с использованием теста Миллера-Рабина  
    }]
    MillerRabinPrimeGeneration[num_Integer, m_] := Module[{n = RandomInteger[{1, num}]},
        While[ProbabilityMillerRabinWitnessQ[n, m] != True, 
           n = RandomInteger[{1, num}]
        ];
        
        Return[n]
    ]
    \end{lstlisting}

    \subsection{Примеры}

      \begin{lstlisting}[language=Mathematica,caption={Пример 1}]
      Timing@MillerRabinPrimeGeneration[10^3, 100]

      {0.109375, 367}
      \end{lstlisting}

  \section{Теорема Диемитко}

    \subsection{1 итерация поиска большего простого числа, чем заданное}

      Вход: {$q$} - простое число \\
      Выход: большее простое число

      \begin{lstlisting}[language=Mathematica,caption={
       Теорема Диемитко (1 итерация)
      }]
    	BigPrimeDiemitko[q_?PrimeQ] := 
    	Module[{n, t = 2 Length@IntegerDigits[q, 2], {$\xi$}, primeQ = False, u = 0, p},
    	  {$\xi$} = RandomReal[{0, 1}];
    	  n = Ceiling[2^(t - 1)/q] + Ceiling[2^(t - 1) {$\xi$}/q];
    	  If[OddQ@n, n = n + 1];
    	  While[! primeQ,
    	    p = (n + u) q + 1;
    	    If[1 == PowerMod[2, p - 1, p] && 1 != PowerMod[2, n + u, p], primeQ = True,
    	    u = u + 2];
    	  ];
    	
    	  Return@p
    	]
      \end{lstlisting}

      \begin{lstlisting}[language=Mathematica,caption={Пример 1}]
      p=Prime[1000000];
      15485863

      q=BigPrimeDiemitkoResults[p];
      223847127598043

      PrimeQ@q;
      True
      \end{lstlisting}

      \begin{lstlisting}[language=Mathematica,caption={Пример 2}]
      p=Prime[100000000];
      2038074743

      q=BigPrimeDiemitkoResults[p];
      4561621090903322441

      PrimeQ@q;
      True
      \end{lstlisting}

    \subsection{Много итераций для генерации числа нужной размерности}
    
      Вход: {$s$} - простое число, {$steps$} - число итераций для получения числа нужной размерности \\
      Выход: большее простое число

      \begin{lstlisting}[language=Mathematica,caption={
	     Теорема Диемитко (много итераций)
      }]
    	BigPrimeDiemitkoComplex[s_, steps_: 5] := Module[{S = s},
    	  Do[S = BigPrimeDiemitko@S, {i, 1, steps}];
      
    	  Return@S;
    	]
      \end{lstlisting}

      \begin{lstlisting}[language=Mathematica,caption={Пример 1}]
      p=Prime[1]];
      2

      q=BigPrimeDiemitkoComplexResults[p];
      368232145871351076619854785594491593935433113140668319772418102971192962388774103259613

      PrimeQ@q;
      True
      \end{lstlisting}

      \begin{lstlisting}[language=Mathematica,caption={Пример 2}]
      p=Prime[10]];
      29

      q=BigPrimeDiemitkoComplexResults[p];
      638663772875710726503109287341875117266575187565363426263952126694444214855630831338080281\
      9335793485340218738695099670054318954313764793371991947313963628189825525449\
      6206580377499246334426253330060341007899760131

      PrimeQ@q;
      True
      \end{lstlisting}

      \begin{lstlisting}[language=Mathematica,caption={Пример 3}]
      p=Prime[1000]];
      7919

      q=BigPrimeDiemitkoComplexResults[p];
      77136673480644556865315768367268314816263106913594119648411208041116063926102788742047941420\
      460510437041054943631262458452187699827924706534869956961247790320954339348028256271194127453\
      196932602593923355735051477022365054953043123683148643201902587671779138858676653809127260758\
      418112399765880889475436811240078372948957427572595574444624516405916208870454959690923772800\
      240906984160048116854335409845660159937506451128215193370881307388134246284948149750620719092\
      771493785478511669201945831056629079

      PrimeQ@q;
      True
      \end{lstlisting}