\chapter{Генерация больших чисел}

  \section{Генерация больших простых чисел}

    Дано: {$digits$} - необходимое число знаков в простом числе \\
    Возвращает: простое число с  {$digits$} знаками

    \begin{lstlisting}[language=Mathematica,caption={
      Генерация больших простых чисел
    }]
      GoodRSAPrimeQ[n_Integer, l_Integer] := PrimeQ[n] && Module[{d = 1, m = (n - 1)/2},
	While[d <= l && ! PrimeQ[m/d], d++];
	Return[ d <= l]
      ]

      RSAPrime[digits_Integer] := Module[{l, cand},
	cand = RandomInteger[{10^(digits - 1), 10^digits - 1}];
	l = 10^Floor[1/30 Log[10.`, cand]];
	If[EvenQ[cand], cand++];
	While[! GoodRSAPrimeQ[cand, l], cand += 2];
	
	Return@cand
      ]

    \end{lstlisting}

  \section{Генерация с использованием теста Миллера-Рабина}

    Дано: {$num$} - интервал для числа {$[1;num]$}, {$m$} - число тестов \\
    Возвращает: простое число {$\in$} {$[1;num]$}

    \begin{lstlisting}[language=Mathematica,caption={
      Генерация с использованием теста Миллера-Рабина  
    }]
      MillerRabinPrimeGeneration[num_Integer, m_] := Module[{n = RandomInteger[{1, num}]},
	While[ProbabilityMillerRabinWitnessQ[n, m] != True, 
	  n = RandomInteger[{1, num}]
	];
      
	Return[n]
      ]
    \end{lstlisting}

  \section{Генерация с использованием малой модифицированной теоремы Ферма}

    \subsection{1 итерация поиска большего простого числа, чем заданное}

      Вход: {$s$} - простое число \\
      Выход: большее простое число

      \begin{lstlisting}[language=Mathematica,caption={
	Генерация с использованием малой модифицированной теоремы Ферма (1 итерация)
      }]
	BigPrimeFermat[s_] := Module[{r, p, prime = False, rabbinResults, step = 0, fermatResult, S = s},
	  While[! prime,
	    r = RandomInteger[{S, 4 S + 2}];
	    If[OddQ@r, r = r + 1];
	  
	    p = S r + 1;
	    rabbinResults = MillerRabinTest[p, 1000];
	    If[rabbinResults, prime = True; Break[], step = step + 1; Continue[]];
	    fermatResult = SmallModifiedFermat[p, r];
	    If[fermatResult, prime = True; Break[], step = step + 1; Continue[]];
	  ];
	  
	  Return@p;
	]
      \end{lstlisting}

    \subsection{Много итераций для генерации числа нужной размерности}
    
      Вход: {$s$} - простое число, {$steps$} - число итераций для получения числа нужной размерности \\
      Выход: большее простое число
  
      \begin{lstlisting}[language=Mathematica,caption={
	Генерация с использованием малой модифицированной теоремы Ферма (много итераций)
      }]
	BigPrimeFermatComplex[s_, steps_: 5] := Module[{S = s},
	  Do[S = BigPrimeFermat@S, {i, 1, steps}];
  
	  Return@S;
	]
      \end{lstlisting}

  \section{Теорема Диемитко}

    \subsection{1 итерация поиска большего простого числа, чем заданное}

      Вход: {$q$} - простое число \\
      Выход: большее простое число

      \begin{lstlisting}[language=Mathematica,caption={
	Теорема Диемитко (1 итерация)
      }]
	BigPrimeDiemitko[q_?PrimeQ] := 
	Module[{n, t = 2 Length@IntegerDigits[q, 2], {$\xi$}, primeQ = False, u = 0, p},
	  {$\xi$} = RandomReal[{0, 1}];
	  n = Ceiling[2^(t - 1)/q] + Ceiling[2^(t - 1) {$\xi$}/q];
	  If[OddQ@n, n = n + 1];
	  While[! primeQ,
	    p = (n + u) q + 1;
	    If[1 == PowerMod[2, p - 1, p] && 1 != PowerMod[2, n + u, p], primeQ = True,
	    u = u + 2];
	  ];
	
	  Return@p
	]
      \end{lstlisting}

    \subsection{Много итераций для генерации числа нужной размерности}
    
      Вход: {$s$} - простое число, {$steps$} - число итераций для получения числа нужной размерности \\
      Выход: большее простое число

      \begin{lstlisting}[language=Mathematica,caption={
	  Теорема Диемитко (много итераций)
      }]
	BigPrimeDiemitkoComplex[s_, steps_: 5] := Module[{S = s},
	  Do[S = BigPrimeDiemitko@S, {i, 1, steps}];
  
	  Return@S;
	]
      \end{lstlisting}