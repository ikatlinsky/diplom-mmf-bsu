% !TEX root = ../DiplomProject.tex

\chapter{LLL-алгоритм и теорема Копперсмита}

	\section{Алгоритм ортогонализации Грама-Шмидта}
		Вход: $v=\{v_1, v_2, \dots, v_n \} \in R^m$ - линейно независимые векторы, $m \ge n$ \\
		Выход: ортогональная система векторов $b^* = \{b^{*}_1, \dots, b^{*}_n \}$

		\begin{lstlisting}[language=Mathematica,caption={
      		Алгоритм ортогонализации Грама-Шмидта	
    	}]
		M[a_List, b_List] := (a.b)/(b.b);

		MultipleProjection[v_, vects_List] := Plus @@ (M[v, #1] #1 &) /@ vects;

		GramSchmidtProcess[v_List] := Module[{b, B, M, m, i, j, n = Length@v},
  			b = Fold[Join[#1, {#2 - MultipleProjection[#2, #1]}] &, {}, v];
  			B = #.# & /@ b;
  			M = Table[M[v[[i]], b[[j]]], {i, 1, n}, {j, 1, i - 1}];
  
  			Return@{b, B, M}
  		]
    	\end{lstlisting}

	\section{LLL-алгоритм}		
		Вход: базис $\{b_1,b_2, \dots, b_n \}$ решетки в $R^m, m \ge n$ \\
		Выход: LLL-приведенный базис решетки

    	\begin{lstlisting}[language=Mathematica,caption={
      		LLL-алгоритм	
    	}]
		LLLAlgorithm[v_List] := 
		 Module[{b, B, M, k, n = Length@v, r, S, m, a, t, V, l},
		  V = v;
		  {b, B, M} = GramSchmidtProcess@v;
		  k = 2;
		  While[k <= n,
		   (*3*)
		   If[Abs@M[[k, k - 1]] > 0.5,
		    (*3.1*)
		    r = Round[M[[k, k - 1]]];
		    (*3.2*)
		    V[[k]] = V[[k]] - r V[[k - 1]];
		    (*3.3*)
		    If[k > 2, 
		     Do[M[[k, j]] = M[[k, j]] - r M[[k - 1, j]], {j, 1, k - 2}]];
		    (*3.4*)
		    M[[k, k - 1]] = M[[k, k - 1]] - r;
		    ];
		   (*4*)
		   If[B[[k]] < (3/4 - M[[k, k - 1]]^2) B[[k - 1]],
		    (*4.1*)
		    m = M[[k, k - 1]]; S = B[[k]] + m^2 B[[k - 1]];
		    M[[k, k - 1]] = m B[[k - 1]]/S; B[[k]] = B[[k - 1]] B[[k]]/S; 
		    B[[k - 1]] = S;
		    (*4.2*)
		    a = V[[k]]; V[[k]] = V[[k - 1]]; V[[k - 1]] = a;
		    (*4.3*)
		    If[k > 2, 
		     Do[a = M[[k, j]]; M[[k, j]] = M[[k - 1, j]]; M[[k - 1, 
		        j]] = a, {j, 1, k - 2}]];
		    (*4.4*)
		    Do[t = M[[j, k]]; M[[j, k]] = M[[j, k - 1]] - m t; M[[j, 
		       k - 1]] = t + M[[k, k - 1]] M[[j, k]], {j, k + 1, n}];
		    (*4.5*)
		    k = Max[2, k - 1],
		    (*5*)
		    If[k > 2,
		     For[l = k - 2, l > 0, l--,
		       If[Abs@M[[k, l]] > 0.5,
		         r = Round[M[[k, l]]];
		         V[[k]] = V[[k]] - r V[[l]];
		         Do[M[[k, j]] = M[[k, j]] - r M[[l, j]], {j, 1, l - 1}];
		         M[[k, l]] = M[[k, l]] - r'
		         ];
		       ];
		     ];
		    k = k + 1;
		    ];
		   ];
		  
	  	  Return@V
  		]
    	\end{lstlisting}

    	\subsection{Примеры}

	    	\begin{lstlisting}[language=Mathematica,caption={
	      		Пример	1
	    	}]
			LLLAlgorithm[{{1, 0, 0, 5}, {0, 1, 0, 5}, {0, 0, 1, 5}}]
			LatticeReduce[{{1, 0, 0, 5}, {0, 1, 0, 5}, {0, 0, 1, 5}}]

			{{-1, 1, 0, 0}, {-1, 0, 1, 0}, {1, 0, 0, 5}}
			{{-1, 1, 0, 0}, {-1, 0, 1, 0}, {1, 0, 0, 5}}
	    	\end{lstlisting}

	    	\begin{lstlisting}[language=Mathematica,caption={
	      		Пример	2
	    	}]
			LLLAlgorithm[{{1, 0, 0, 1345}, {0, 1, 0, 35}, {0, 0, 1, 154}}]
			LatticeReduce[{{1, 0, 0, 1345}, {0, 1, 0, 35}, {0, 0, 1, 154}}]

			{{0, 9, -2, 7}, {1, 1, -9, -6}, {1, -3, -8, 8}}
			{{0, 9, -2, 7}, {1, 1, -9, -6}, {1, -3, -8, 8}}
	    	\end{lstlisting}

	\section{Алгоритм Копперсмита}
		Вход: $f$ - полином, $NN=p q$ - составное число, $p$ и $q$ - простые \\
		Выход: малые корни полинома $f$

		\begin{lstlisting}[language=Mathematica,caption={
      		Алгоритм Копперсмита
    	}]
		g[f_, x_, u_, v_, NN_, m_] := Expand@(NN^u x^v f[x]^(m - u));

		Coppersmith[f_, NN_] := 
		 Module[{B = 1/2, E, D = Exponent[f[x], x], m, t, n, X, 
		   G, j, i, B = {}, L, a, GG, list, res = {}, H},
		  E = 1/7 B;
		  m = Ceiling[B^2/D/E];
		  t = Floor[D m (1/B - 1)];
		  n = m D + t;
		  X = Ceiling[1/2 NN^(B^2/D - E)];
		  G = Array[0 &, {D, m}];
		  
		  For[j = 0, j <= D - 1, j++,
		   For[i = 1, i <= m, i++,
		     G[[j + 1, i]] = g[f, x, i, j, NN, m];
		     ];
		   ];
		  
		  H = NestList[x # &, f[x]^m, t - 1];
		  G = Flatten@Union[G, H];
		  
		  G = G /. x -> x X;
		  Table[B = Append[B, Table[Coefficient[G[[j]], x, i], {i, 0, n - 1}]];, {j, 
		    1, Length@G}];
		  
		  L = LatticeReduce@B;
		  L = Sort[L, Sqrt[#1.#1] <= Sqrt[#2.#2] &];
		  a = First@L;
		  
		  list = Solve[GG == 0, x, Integers];
		  Table[If[GCD[NN, IntegerPart[f[list[[i, 1, 2]]]]] >= NN^B, 
		    res = Append[res, list[[i, 1, 2]]]], {i, 1, Length@list}];
		  
		  Return@res;
		]
    	\end{lstlisting}

    	\subsection{Примеры}

	    	\begin{lstlisting}[language=Mathematica,caption={
	      		Пример	1
	    	}]
			f[x_] := Expand@((x - 1) (x - 2) (x - 3) (x - 4) (x - 40))
			NN = 12458;
			Coppersmith[f, NN]

			{1, 2}
	    	\end{lstlisting}

	    	\begin{lstlisting}[language=Mathematica,caption={
	      		Пример	2
	    	}]
			p = RSAPrime[10]
			q = RSAPrime[10]
			n = p q
			f2[x_] := Expand@((x - 1) (x - 2) (x - 3) (x - 4) (x - 40) (x - 40) (x - 156) (x - 998))
			Coppersmith[f2, n]

			9292927559
			7642228967
			71018680179622401553
			{1, 2, 3, 4}
	    	\end{lstlisting}

	    	\begin{lstlisting}[language=Mathematica,caption={
	      		Пример	3
	    	}]
			p = RSAPrime[50]
			q = RSAPrime[50]
			n = p q
			f3[x_] := Expand@((x - 1) (x - 2) (x - 3) (x - 4) (x - 5) (x - 40) (x - 40) (x - 998) x)
			Coppersmith[f3, n]

			97782689110580772186708411304526221212574986151967
			65956799531101932713749780654432033005384499563983
			644943322327863993696069981091221157509595595720458070665516084380384623599211\
			1377402958574477804561
			{0, 0, 1, 2, 3, 4, 5, 40, 40, 998}
	    	\end{lstlisting}

	\section{Упрощенная проблема RSA}
		Вход: {$m^e, m', |m-m'| \le N^\frac{1}{e}$} \\   
  		Выход: {$m \in Z_N$}  

		\subsection{Начальные параметры}

			\begin{lstlisting}[language=Mathematica,caption={
	      		Начальные параметры
	    	}]
			p = RSAPrime[40]
			q = RSAPrime[40]
			n = p q
			e = 7
			m = 15896
			c = PowerMod[M, e, n]

			9622652790050554336371812544520812701891
			9263560181542628546685268518468809256383
			89140023226722394226112879285436419041513765944205436610065094651378340767920253
			7
			15896
			256457248648114000263869628416
	    	\end{lstlisting}

    	\subsection{Полином аппроксимации}

    		\begin{lstlisting}[language=Mathematica,caption={
	      		Полином аппроксимации
	    	}]
			m' = 15800;

			relaxedProblemPolynom[x_] := (m' + x)^e - c	
	    	\end{lstlisting}

    	\subsection{Решение}

    		\begin{lstlisting}[language=Mathematica,caption={
	      		Решение
	    	}]
			Coppersmith[relaxedProblemPolynom, n]
			96

			m = m' + 96 = 15896
	    	\end{lstlisting}

	\section{Атака Франклина-Райтера}
  		Вход: {$c = m^e (mod \: N)$}, {$c^{'} = (m+r)^e (mod \: N)$}, {$|r| \le N^{1/e^{2}}$} \\
  		Выход: $m$

		\subsection{Функция зависимости сообщений и начальные параметры}

			\begin{lstlisting}[language=Mathematica,caption={
	      		Функция зависимости сообщений и начальные параметры
	    	}]
			FrankReiterPolynom[a_, b_, m_] := a m + b

			p = RSAPrime[20]
			34252875648544607507

			q = RSAPrime[20]
			89415793685815988603

			n = p q
			3062748062136175151514402826557820242721

			e = 3;
			d = GenerateD[e, p, q]
			2041832041424116767593822771482306431075
	    	\end{lstlisting}

    	\subsection{$m1$ и $m2$ - сообщения для шифрования}

    		\begin{lstlisting}[language=Mathematica,caption={
	      		m1 и m2 - сообщения для шифрования
	    	}]
			a = 1;
			b = 3;
			m1 = 6;

			m2 = FrankReiterPolynom[a, b, m1]
			9
	    	\end{lstlisting}

    	\subsection{$c1$ и $c2$ - шифротексты}

    		\begin{lstlisting}[language=Mathematica,caption={
	      		Шифротексты
	    	}]
			c1 = RSAEncodeNumber[m1, e, n]
			216

			c2 = RSAEncodeNumber[m2, e, n]
			729
	    	\end{lstlisting}

    	\subsection{Решение}

    		\begin{lstlisting}[language=Mathematica,caption={
	      		Решение
	    	}]
			FrankReitersAttack[c1_, c2_, n_, a_, b_] := Module[{f, g},
			  f = Mod[b (c2 + 2 a^3 c1 - b^3), n];
			  g = Mod[a (c2 - a^3 c1 + 2 b^3), n];
			  
			  Return@First@(x /. Solve[g x == f, x, Modulus -> n])
			]

			FrankReitersAttack[c1, c2, n, a, b]
			6
			m1 = 6
			m2 == FrankReiterPolynom[a, b,m1]
	    	\end{lstlisting}

	\section{Факторизация при известной аппроксимации}
		Вход: {$ N = p q, p': |p-p'| \le N^\frac{1}{4} $} \\   
  		Выход: {$p$}  

  		\subsection{Пример 1}

	  		\subsubsection{Инициализация параметров}

	  			\begin{lstlisting}[language=Mathematica,caption={
		      		Инициализация параметров
		    	}]
				p = 25785228123955207329992023669947784861995718349736955586026199192716446882\
				89101973195147518151530813;

				q = 10668673279670479640047334548171542196218037195733888904249796125333457239\
				565191307619576134523433776492344849818098672379446634699745951851743235009385\
				642278100294010559055776215425912251443375612531;

				X = 158823917060391733218296424878747562818233855499155999;

				n = p q;

				P = q - X + 1;
		    	\end{lstlisting}

	    	\subsubsection{Решение}

	    		\begin{lstlisting}[language=Mathematica,caption={
		      		Решение
		    	}]
				FactorizationPolynom[x_] := x + P

				sol=Coppersmith[FactorizationPolynom, n];
				{158823917060391733218296424878747562818233855499155998}

				(First@sol + P) == q
				True

				(n/(First@sol + P) == p
				True
		    	\end{lstlisting}

    	\subsection{Пример 2}

	  		\subsubsection{Инициализация параметров}

	  			\begin{lstlisting}[language=Mathematica,caption={
		      		Инициализация параметров
		    	}]
				p = RSAPrime[150];
				q = RSAPrime[200];
				X = 4487454554185521545415451215421212154512121524541;
				n = p q;
				P = q - X + 1;
		    	\end{lstlisting}

	    	\subsubsection{Решение}

	    		\begin{lstlisting}[language=Mathematica,caption={
		      		Решение
		    	}]
				FactorizationPolynom[x_] := x + P

				sol=Coppersmith[FactorizationPolynom, n]
				{4487454554185521545415451215421212154512121524540}

				(First@sol + P) == q
				True

				(n/(First@sol + P) == p
				True
		    	\end{lstlisting}

