% !TEX root = ../DiplomProject.tex

\chapter{Основные алгоритмы криптосистемы RSA}

	\section{Генерация числа e}
		Дано: $p$, $q$ - простые числа \\
		Возвращает: часть открытого ключа - $e$

		\begin{lstlisting}[language=Mathematica,caption={
      		Генерация числа e	
    	}]
		GenerateE[p_Integer, q_Integer] := Module[{res, n = p q, phi = (p - 1) (q - 1)}, 
			res = RandomInteger[{p, n}]; 
			While[GCD[res, phi] != 1, res = RandomInteger[{p, n}]];

			Return@res
		]
    	\end{lstlisting}

    	\subsection{Примеры}

    	\begin{lstlisting}[language=Mathematica,caption={
      		Пример	
    	}]
		n = p q;

		1855408329482759304710972417348746\
    60568239845230771926992813068239006\
		462676943335830802004690371432925069083067363684541;

		e = GenerateE[p, q];

		16338706635166137769682750826265830\
    79087602267653090223551409078842007\
		2913861142884282761561489568870311006453494838051
    	\end{lstlisting}

	\section{Генерация числа d}
		Дано: $p$, $q$ - простые числа \\
		Возвращает: секретный ключ $d$

		\begin{lstlisting}[language=Mathematica,caption={
      		Генерация числа d	
    	}]
    	GenerateD[e_Integer, p_Integer, q_Integer] := PowerMod[e, -1, (p - 1) (q - 1)]
    	\end{lstlisting}

    	\subsection{Примеры}

    	\begin{lstlisting}[language=Mathematica,caption={
      		Пример	
    	}]
		d = GenerateD[e, p, q];

		91589468421478452278600226765331088984908\
    1812083439831186947346958122837148235\
		70501597775572223951191041670828593718731
    	\end{lstlisting}

	\section{Шифрование числа}
		Дано: $num$ - число для шифрования, $e$ и $n$ - открытый ключ \\
		Возвращает: шифр

		\begin{lstlisting}[language=Mathematica,caption={
      		Шифрование заданного числа	
    	}]
    	RSAEncodeNumber[num_Integer, e_Integer, n_Integer] := 
    	PowerMod[num, e, n] /; num < n
    	\end{lstlisting}

    	\subsection{Примеры}

    	\begin{lstlisting}[language=Mathematica,caption={
      		Пример	
    	}]
		c = RSAEncodeNumber[123, e, n];

		10205861680659800219013147789801975997759\
    2438305516544926415510748598280467606\
		982364146907305716227019845783824533161550
    	\end{lstlisting}

	\section{Дешифрование числа}
		Дано: $num$ - число для расшифровки, $d$ - секретный ключ, $n$ - открытый ключ \\
		Возвращает: расшифрованное число

		\begin{lstlisting}[language=Mathematica,caption={
      		Дешифрование числа	
    	}]
		RSADecodeNumber[num_Integer, d_Integer, n_Integer] := PowerMod[num, d, n]
    	\end{lstlisting}

    	\subsection{Примеры}

    	\begin{lstlisting}[language=Mathematica,caption={
      		Пример	
    	}]
		RSADecodeNumber[c, d, n]

		123
    	\end{lstlisting}

	\section{Алгоритм перевода строки в числовую последовательность}
		Дано: $text$ - строка, переводимая в числа, $n$ - часть открытого ключа \\
		Возвращает: последовательность чисел

		\begin{lstlisting}[language=Mathematica,caption={
      		Перевод строки в числовую последовательность
    	}]
		ConvertString[str_String] := Fold[Plus[256 #1, #2] &, 0, ToCharacterCode[str]];

		StringToList[text_String, n_Integer] :=	
			Module[{blockLength = Floor[N[Log[256, n]]], strLength = StringLength[text]},    
       		ConvertString /@ Table[StringTake[text, 
              {i, Min[strLength, i + blockLength - 1]}],
           {i, 1, strLength, blockLength}]
     	] /; n > 256
    	\end{lstlisting}

    	\subsection{Примеры}

    	\begin{lstlisting}[language=Mathematica,caption={
      		Пример	
    	}]
		codedText = StringToList["This is a message", n];

		28722506059135649064412911394457431140197
    	\end{lstlisting}

	\section{Алгоритм перевода числовой последовательности в строку}
		Дано: $l$ - последовательность, преобразуемая в строку \\
		Возвращает: строка

		\begin{lstlisting}[language=Mathematica,caption={
      		Перевод числовой последовательности в строку
    	}]
		MakeList[0] = {};

		MakeList[num_Integer] := Append[MakeList[Quotient[num, 256]], Mod[num, 256]]

		ConvertNumber[num_Integer] := FromCharacterCode /@ MakeList[num]

		ListToString[l_List] := StringJoin[ConvertNumber /@ l]
    	\end{lstlisting}

    	\subsection{Примеры}

    	\begin{lstlisting}[language=Mathematica,caption={
      		Пример	
    	}]
		ListToString[codedText]

		"This is a message"
    	\end{lstlisting}

	\section{Алгоритм шифрования текста}
		Дано: $text$ - строка для шифрования, $e$ и $n$ - открытые ключи \\
		Возвращает: криптотекст

		\begin{lstlisting}[language=Mathematica,caption={
      		Алгоритм шифрования текста
    	}]
		RSAEncodeString[text_String, e_Integer, n_Integer] := 
 			RSAEncodeNumber[#, e, n] & /@ StringToList[text, n] /; n > 256
    	\end{lstlisting}

    	\subsection{Примеры}

    	\begin{lstlisting}[language=Mathematica,caption={
      		Пример	
    	}]
		c = RSAEncodeString["A secret message", e, n];

		74899466982326048953779595360320106236\
    869217435766292239410245927328265139809\
		552597334992459785668849987607462481180285
    	\end{lstlisting}

	\section{Алгоритм дешифрования текста}
		Дано: $l$ - шифр строки, $d$ - секретный ключ, $n$ - открытый ключ \\
		Возвращает: расшифрованная строка

		\begin{lstlisting}[language=Mathematica,caption={
      		Алгоритм шифрования текста
    	}]
		RSADecodeString[l_List, d_Integer, n_Integer] := 
 			ListToString[RSADecodeNumber[#, d, n] & /@ l]
    	\end{lstlisting}

    	\subsection{Примеры}

    	\begin{lstlisting}[language=Mathematica,caption={
      		Пример	
    	}]
		RSADecodeString[c, d, n];

		"A secret message"
    	\end{lstlisting}

