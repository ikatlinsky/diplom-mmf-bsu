% !TEX root = DiplomProjectPresentation.tex

\begin{frame}{Актуальность}
	\begin{center}
		Самой известной и применимой на практике криптосистемой с открытым ключом является криптосистема RSA. \\

		\begin{itemize}
			\item Рост вычислительной мощности.
			\item Усовершенствование старых алгоритмов криптоатак.
			\item Создание новых подходов к криптоатакам.			
		\end{itemize}

		$\Downarrow$

		Поэтому важным является
		\begin{itemize}
			\item Построение стойких ключей - получение больших простых чисел.
			\item Изучение криптостойкости шифра.					
		\end{itemize}
	\end{center}
\end{frame}

\begin{frame}{Цели и задачи}	
	\begin{importantblock}{Цель}{}{bg=blue}
		Построение больших простых чисел и исследование криптостойкости шифра RSA.	
	\end{importantblock}	

	\begin{block}{Задача 1}
		Разобрать способы построения простых чисел.	
	\end{block}	

	\begin{block}{Задача 2}
		Провести сравнение существующих алгоритмов факторизации.
	\end{block}

	\begin{block}{Задача 3}
		Провести криптоанализ шифра RSA.
	\end{block}

	\begin{block}{Задача 4}
		Реализовать изученные алгоритмы в пакете Mathematica.
	\end{block}
		
\end{frame}