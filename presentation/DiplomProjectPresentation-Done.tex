% !TEX root = DiplomProjectPresentation.tex

\begin{frame}{Построение больших простых чисел}
	\begin{block}{Методы}
		\begin{itemize}
			\item Тест Миллера-Рабина - тест чисел на простоту 
			\item Теорема Диемитко - построение простых чисел        
		\end{itemize}	
	\end{block}

	\begin{importantblock}{Теорема Диемитко}{}{bg=blue}
		Пусть {$n = q R + 1$}, где {$q$} - простое число, {$R$} - четное, {$R < 4(q + 1)$}. Если найдется {$a < n$}:
      
			\begin{enumerate}
			 \item {$a^{n - 1} \equiv 1(mod \: n)$}
			 \item {$a^{\frac{n - 1}{q}} \not\equiv 1(mod \: n)$}
			\end{enumerate}
			
      то {$n$} – простое число.
	\end{importantblock}
\end{frame}

\begin{frame}{Построение больших простых чисел}
		\begin{center}
			\begin{block}{Сравнение методов}
			\centering
			\resizebox{\linewidth}{!}{
			\begin{tabular}{cccc}	  
			    \toprule  
			    \textbf{Метод} & \textbf{Вероятностный} & \textbf{Точный} & \textbf{Полиномиальный} \\
			    \midrule
			    Миллер-Рабин        & +           & -                & +             \\
			    Диемитко            & -           & +                & +             \\		    
			    \bottomrule
			\end{tabular} 	
			}
		\end{block} \vspace{0.3cm}

		{\LARGE Выводы } \vspace{0.3cm}

		Алгоритм, основанный на теореме Диемитко, больше подходит для криптосистемы RSA

		\begin{itemize}
			\item Строит большие простые числа, а не тестирует
			\item Полиномиальный
			\item Всегда дает точный результат
		\end{itemize}
	\end{center}	
\end{frame}

\begin{frame}{Алгоритмы факторизации}
	\begin{block}{Экспоненциальные алгоритмы}
		\begin{itemize}
			\item Перебор возможных делителей - {$O(N\textsuperscript{{$\frac{1}{2}$}})$}
			\item Метод факторизации Ферма - {$O(N\textsuperscript{{$\frac{1}{4}$}})$}
	        \item {$\rho$}-алгоритм Полларда - {$O(N\textsuperscript{{$\frac{1}{4}$}})$}	        
    	\end{itemize}
	\end{block}

	\begin{center}
		$L_N(\alpha,c)=O(exp((c+o(1))(logN)^\alpha(loglogN)\textsuperscript{{$1-\alpha$}}))$
	\end{center}

	\begin{block}{Субэкспоненциальные алгоритмы}
		\begin{itemize}
			\item Алгоритм Диксона - {$L_N(\frac{1}{2}, 2\sqrt{2})$}
	        \item Метод непрерывных дробей - {$L_N(\frac{1}{2}, \sqrt{2})$}
	        \item Метод квадратичного решета - {$L_N(\frac{1}{2}, 1)$}
	        \item Метод решета числового поля - {$L_N(\frac{1}{3}, (\frac{64}{9})^\frac{1}{3})$} 
    	\end{itemize}
	\end{block}
\end{frame}

\begin{frame}{Алгоритмы факторизации}
	\begin{center}
		{\LARGE Выводы }

		\begin{itemize}
			\item Выбор алгоритма факторизации - нетривиальная задача.
			\item Выбор оптимального алгоритма зависит от многих параметров - тип и вычислительная скорость.
			\item Важным фактором является структура числа, подлежащего факторизации.
			\item При выборе алгоритма стоит учитывать размерность числа.
			\item Эффективность алгоритмов факторизации опирается не только на вычислительные мощности, но и на достижения в области математики и теории чисел.
		\end{itemize}
	\end{center}
\end{frame}

\begin{frame}{Криптоанализ RSA}
	\begin{center}

		\begin{block}{Подходы к криптоанализу}
			\begin{itemize}
		        \item Разложение $N$ на простые множители $p$ и $q$.
		        \item Определение $\phi(N)$ без $p$ и $q$.
		        \item Определение $d$ без $\phi(N)$.
    		\end{itemize}
		\end{block}	 		

		$\Downarrow$

	 	\begin{block}{Реализация подходов}
	 		\begin{itemize}
		        \item Простой перебор.
		        \item Математический подход.
    		\end{itemize}
	 	\end{block}	

	 	\begin{block}{Примеры взломов}
	 		\begin{itemize}
		        \item Малая экспонента.
		        \item Перебор возможных открытых текстов.
		        \item Общий модуль.
    		\end{itemize}
	 	\end{block}	
    
	\end{center}
\end{frame}

\begin{frame}{Криптоанализ RSA}			

	\changefontsizes{10pt}

	\begin{importantblock}{Теорема Копперсмита}{}{bg=blue}
		Пусть $N$ - целое число, факторизация которого неизвестна и которое имеет делитель {$b \ge N^\beta, 0 < \beta \le 1$}. Пусть {$f(x)$} - 
		унитарный многочлен от одной переменной степени {$\delta$} и {$c \ge 1$}, тогда можно найти все решения {$x_0$} уравнения
		    \begin{equation}
		      f(x) \equiv 0(mod \: b), |x_0| \le c N^{\frac{\beta^2}{\delta}}
		    \end{equation}
		за время {$O(c \delta^5 log^9 N)$}.
	\end{importantblock}	 
	
	\begin{block}{Факторизация}
		Дано: $N = p \: q$ \\
		Найти: $p$
	\end{block}	

	\begin{block}{Факторизация + аппроксимация}
		Дано: $N = p \: q$, $p': |p-p'| \le N^\frac{1}{4}$ \\
		Найти: $p$
	\end{block}		 

	\changefontsizes{13pt}
	
\end{frame}

\begin{frame}{Криптоанализ RSA}	
	\begin{center}
		{\LARGE Атаки }

		\begin{itemize}
			\item Упрощенная проблема RSA \\
				Дано: {$m^e, m', |m-m'| \le N^\frac{1}{e}$} \\   
  				Найти: {$m \in Z_N$} 

			\item Атака Франклина-Райтера \\
				$m_2 = f(m_1)(mod \: N)$ \\
				$f = Ax+b, \: e = 3$
			\item Расширенная атака Хастаадта \\
				$c_i=(i \: 2^H + m)^e(mod \: N_i), \: g_i(x) = (i \: 2^H + x)^e(mod \: N_i) - c_i$
			\item Атака Винера \\
				$\frac{e}{N}$ - в непрерывную дробь \\
				$((m)^e)^d = m (mod \: N)$
		\end{itemize}
	\end{center}	
\end{frame}

\begin{frame}{Криптоанализ RSA}	
	\begin{center}
		{\LARGE Выводы }

		\changefontsizes{8pt}

		\begin{itemize}
			\item Благодаря теореме Копперсмита можно найти решение полиномиального уравнения в кольце вычетов при условии, что решение мало. Алгоритм поиска таких решений основан на построении решетки, а именно на LLL-алгоритме.
	      	\item Частичное раскрытие ключа шифра может привести к полному раскрытию ключа и, как следствие, взлому шифра RSA. Это вытекает из теоремы Копперсмита и примеров ее использования.
	      	\item Не стоит использовать малую шифрующую экспоненту в шифре RSA, поскольку это может привести к взлому с помощью атаки Винера или китайской теоремы об остатках.
	      	\item Секретный ключ $d$ необходимо выбирать среди чисел не меньших, чем $\frac{1}{3} N^{\frac{1}{4}}$
	      	\item Необходимо осторожно пользоваться схемами дополнения в шифре RSA, поскольку это может привести к взлому с помощью атак Хастадта и Франклина-Райтера.
		\end{itemize}

		\changefontsizes{13pt}
	\end{center}	
\end{frame}

\begin{frame}{Реализация алгоритмов}			

	\begin{block}{Использованная среда}
		Mathematica
	\end{block}	 
	
	\begin{block}{Реализовано}
		\begin{itemize}
	        \item Тесты чисел на простоту
	        \item Генерация больших простых чисел
	        \item Алгоритмы факторизации
	        \item Алгоритмы шифрования и дешифрования
	        \item LLL-алгоритм + теорема Копперсита
	        \item Криптоатаки на шифр RSA
		\end{itemize}
	\end{block}	 
	
\end{frame}