% !TEX root = DiplomProjectPresentation.tex

\begin{frame}{Систематизация информации}
	\begin{block}{1}
		Изучены известные результаты.
	\end{block}

	\begin{block}{2}
		Разобраны и изучены принципы криптосистемы RSA.
	\end{block}

	\begin{block}{3}
		Произведен криптоанализ RSA.
	\end{block}
\end{frame}

\begin{frame}{Известные результаты}
	\begin{center}
		Табличка со взломами RSA ключей!?	 
	\end{center}
\end{frame}

\begin{frame}{Принципы криптосистемы RSA}
	\begin{center}
		\begin{itemize}
			\item Криптосистемы с открытым ключом.
	        \item Простые числа - тесты и генерация.
	        \item Факторизация чисел.
	        \item Алгоритмы шифрования, дешифрования, создание ключей.
    	\end{itemize}
	\end{center}
\end{frame}

\begin{frame}{Криптоанализ RSA}
	\begin{center}

		\begin{block}{Подходы к криптоанализу}
			\begin{itemize}
		        \item Разложение $N$ на простые множители $p$ и $q$
		        \item Определение $\phi(N)$ без $p$ и $q$
		        \item Определение $d$ без $\phi(N)$
    		\end{itemize}
		\end{block}	 		 
    
	\end{center}
\end{frame}

\begin{frame}{Криптоанализ RSA}			

	\begin{block}{Теорема Копперсмита}
		Взлом криптосистемы RSA при известной аппроксимации
	\end{block}	 
	
	\begin{block}{Факторизация}
		Дано: $N = p \: q$ \\
		Найти: $p$
	\end{block}
	

	\begin{block}{Факторизация + аппроксимация}
		Дано: $N = p \: q$, $p': |p-p'| \le N^\frac{1}{4}$ \\
		Найти: $p$
	\end{block}		 
	
\end{frame}

\begin{frame}{Реализация алгоритмов}			

	\begin{block}{Использованная среда}
		Mathematica
	\end{block}	 
	
	\begin{block}{Реализовано}
		\begin{itemize}
	        \item Тесты чисел на простоту
	        \item Генерация больших простых чисел
	        \item Алгоритмы факторизации
	        \item Алгоритмы шифрования и дешифрования
	        \item LLL-алгоритм + теорема Копперсита и их приложения
		\end{itemize}
	\end{block}	 
	
\end{frame}