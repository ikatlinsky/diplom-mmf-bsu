% !TEX root = DiplomProjectPresentation.tex

\begin{frame}{Результаты}
	\begin{center}
		\begin{enumerate}
			\item Разобраны, сравнены и реализованы алгоритмы получения простых чисел - тест Миллер-Рабина, теорема Диемитко. \vspace{0.5cm}
			\item Разобраны, сравнены и реализованы алгоритмы факторизации - экспоненциальные и субэкспоненциальные. \vspace{0.5cm}
			\item Проведен криптоанализ шифра RSA, указаны слабые места при создании ключей для шифра. \vspace{0.5cm}
			\item Реализованы криптоатаки на шифр RSA.
		\end{enumerate}
	\end{center}
\end{frame}

\begin{frame}{Заключение}

	\changefontsizes{8pt}

	\begin{center}		

		\begin{block}{RSA}
			\begin{enumerate}
				\item Распространенная криптосистема
				\item Криптостойкая, при выполнении определенных условий
				\item Легко реализуется
				\item Алгоритмы создания ключей, шифрования и дешифрования работаю за полиномиальное время
			\end{enumerate}	
		\end{block}

		\begin{block}{Рекомендации при использовании шифра RSA}
			\begin{enumerate}
				\item Экспонента $e$ должна быть достаточно большим числом
				\item Минимальный размер числа $n$ - 768 бит, рекомендуется от 1024 до 2048
				\item Простые числа $p$ и $q$ : $N=pq$ должны быть одинаковой длины, разность чисел $p - q$ не должна быть маленькой
				\item Не рекомендуется использование общих модулей $N$ разными абонентами
				\item Не рекомендуется использовать одинаковые экспоненты $e$
			\end{enumerate}	
		\end{block}
	\end{center}

	\changefontsizes{13pt}

\end{frame}
